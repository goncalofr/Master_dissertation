\documentclass[defaultstyle,10pt,master,Helvetica]{thesis}
% Helvetica is a similar font to Arial, with small differences.

%--------------------------------------------------------------
%debug features - some may not be included in the final document
%obter informacao de debug acerca de parêntesis mal fechados
\tracinggroups=1
%\usepackage{showframe}    %show page dimensions, to visually confirm page layout
%\usepackage{showlabels}     %show labels in the printed pdf, for easy finding.
\usepackage{todo} %keep track of things to do in the document
%http://tex.stackexchange.com/questions/55618/visual-debugging-of-lengths-in-paragraphs-and-environments
%\usepackage{layouts}  - it is producing a compilation warning
%\usepackage{lua-visual-debug} only compatible with LuaTex
%--------------------------------------------------------------


%% Packages
\typeout{}
\typeout{--------------------------------------------------------------}
\typeout{ +---+ Thesis Template                            }
\typeout{ +---+      Version 2.0, August 2011                         }
\typeout{ +---+  for Instituto Superior Tecnico (IST),                 }
\typeout{ +---+  Universidade Tecnica de Lisboa                         }
\typeout{ * Using Thesis Style form Pedro Tomas                                }
\typeout{ * Created to write Dissertations                             }
\typeout{ * Conforms with IST Master Degree format and with most important packages setup        }
\typeout{ * Should conform with IST PhD Degree format (not verified)   }
\typeout{                                                              }
\typeout{ AUTHOR: Miguel Amador and Joao Marques                                          }
\typeout{ MODIFIED BY: Goncalo Andre                                   }
\typeout{                                                              }
\typeout{ Important: Use all files in the archive, since this is based in all them. Modify dummy files at wish.                                        }
\typeout{--------------------------------------------------------------}
\typeout{}

% Defines an additional alphabet... not required in most cases
% ------------------------------------------------------------
% \DeclareMathAlphabet{\mathpzc}{OT1}{pzc}{m}{it}

% PACKAGE babel:
% ---------------
% The 'babel' package may correct some hyphenisation issues of latex. 
% However in most situations it is not required.
\usepackage[portuguese]{babel}
%\usepackage[fixlanguage]{babelbib} - this does not support ieee bibtex style

% PACKAGE fontenc:
% -----------------
% chooses T1-fonts and allows correct automatic hyphenation.
% from: http://tex.stackexchange.com/questions/664/why-should-i-use-usepackaget1fontenc/677#677
% If you don't use \usepackage[T1]{fontenc},
%   -Words containing accented characters cannot be automatically hyphenated,
%   -You cannot properly copy-and-paste such words from the output (DVI/PS/PDF)
%   -Characters like the pipe sign, less than and greater sign give unexpected results in text.
\usepackage[T1]{fontenc}
% http://tex.stackexchange.com/questions/44694/fontenc-vs-inputenc
% allows the user to input accented characters directly form the keyboard
\usepackage[latin1]{inputenc}
\usepackage{lmodern}

%package verbatim - needed for comment environment
\usepackage{verbatim}

% Package ulem (underlining for emphasis).
% the ulem package provides various types of underlining that can
% stretch between words and be broken across lines.
\usepackage{ulem} % Allows the use of other text emphatizer commands
\normalem % () defines \emph{} back to italic, instead of underline (reverts change from ulem package). 
\raggedbottom %declaration makes all pages the height of the text on that page. No extra vertical space is added. The \flushbottom declaration makes all text pages the same height, adding extra vertical space when necessary to fill out the page.

% PACKAGE date time:
% -----------------
% Lets you alter the format of the date that \today returns.
\usepackage{datetime}
\newdateformat{todaythesis}{%
\monthname[\THEMONTH]  \THEYEAR}

% PACKAGE latexsym:
% -----------------
% Defines additional latex symbols. May be required for thesis with many math symbols.
\usepackage{latexsym}

% PACKAGE amsmath, amsthm, amssymb, amsfonts:
% -------------------------------------------
% This package is typically required. Among many other things it adds the possibility
% to put symbols in bold by using \boldsymbol (not \mathbf); defines additional 
% fonts and symbols; adds the \eqref command for citing equations. I prefer the style
% "(x.xx)" for referering to an equation than to use "equation x.xx".
\usepackage{amsmath, amsthm, amssymb, amsfonts, amsbsy}
%used for the \vrectangleblack in serial terminal examples
\usepackage{stix}

%needed for tikz
\usepackage{standalone} %needed because of floating tickz pictures
\usepackage[usenames,dvipsnames]{xcolor}

% PACKAGE multirow, colortbl, longtable:
% ---------------------------------------
% These packages are most usefull for advanced tables. The first allows to join rows 
% throuhg the command \multirow which works similarly with the command \multicolumn
% The second package allows to color the table (both foreground and background)% The fourth package is only required when tables extend beyond the length of one page;
% The fourth package is only required when tables extend beyond the length of one page;
% with compatibilities with the tabular environment. The third allows the definitions of landscape pages, allowing the use of a different orientation for wider graphics or tables. See package documentation to see the implementation.
\usepackage{multirow}
\usepackage{colortbl}
\usepackage{pdflscape}
%\usepackage{xtab}
\usepackage{longtable}
%\usepackage{afterpage} %needed to avoid longtable page breaks
\usepackage{tabulary}
\usepackage{threeparttable}
%\usepackage{tablefootnote}

% PACKAGE graphics, epsfig, subfigure, caption:
% ---------------------------------------------
% Packages for figures... well you will certainly need these packages, with the exception
% of the 'caption' package. This only allows to define extra caption options.
% Notice that subfigure allows to place figures within figures with its own caption. It
% should be avoided to create an eps file with subfigures. That will mean that you won't be 
% able to reference those subfigures. Instead create an EPS file (the only graphics format supported
% by latex) for each of the subfigures and then use the command \subfigure (see below).
\usepackage{adjustbox}
\usepackage{graphics}
\usepackage{graphicx}
\usepackage{epsfig}
%\usepackage[hang,small,bf]{subfigure} - obsolete package, replace by subcaption
%\usepackage[footnotesize,bf,center]{caption}
\usepackage{dcolumn}
\usepackage{bm}
\usepackage{booktabs}
\usepackage{rotating}
\usepackage{multirow}

\usepackage[font=small,labelfont=bf,textfont=normalfont,justification=centering]{caption}
\usepackage[font=small,labelfont=bf,textfont=normalfont]{subcaption}
%http://tex.stackexchange.com/questions/91566/syntax-similar-to-centering-for-right-and-left/91580#91580?newreg=ab93b1a4f79a4315b773d13d37d37678
%easilly define horizontal orientation of images.
%\usepackage[export]{adjustbox} %causing conflicts - had to find a way around

%include pdf pages in document
\usepackage{pdfpages}

% Improves the interface for defining floating objects such as figures and tables
\usepackage{float}

% PACKAGE listings
% ------------------------------
% This package is used to list source code
\usepackage{listings}

%this package is used for quotations
\usepackage{dirtytalk}

% PACKAGE algorithmic, algorithm
% ------------------------------
% These packages are required if you need to describe an algorithm.
% \usepackage{algorithmic}
% \usepackage[chapter]{algorithm}

% PACKAGE natbib/cite
% -------------------
% The two packages are not compatible, and you should use one of the two. Notice however that the
% IEEE BiBTeX stylesheet is imcompatible with the natbib package. If using the IEEE format, use the 
% cite package instead
\usepackage[square,numbers,sort&compress]{natbib}
%\usepackage{cite}

% PACKAGE acronyum
% -----------------
% This package is most useful for acronyms. The package guarantees that all acronyms definitions are 
% given at the first usage. IMPORTANT: do not use acronyms in titles/captions; otherwise the definition 
% will appear on the table of contents.
\usepackage[printonlyused]{acronym}
\usepackage[titletoc,title,header]{appendix} %modify the titles of appendixes
\usepackage{etoc}       %http://tex.stackexchange.com/questions/228729/how-to-hide-chapter-numbering-in-table-of-contents
%Set page numbering by chapter. The noauto option tells the package not to set pagenumbers: this has to be done by the user
\usepackage[noauto]{chappg}

% PACKAGE extra_functions VER COMO DEVE SER
% -----------------
% My Personal package: defines the following commands:
% \fancychapter{chaptername) -> Prints a fancier chapter (you can also use the fancychapter package for this)
% \hline{width} -> use for a replacement of the \hline command
% \Mark1, \Mark2, \Mark3, ...
\usepackage{extra_functions}


% PACKAGE hyperref
% -----------------
% Set links for references and citations in document
% Some MiKTeX distributions have faulty PDF creators in which case this package will not work correctly
% Long live Linux :D
% This package should come to the end of the includes, because it redefines many macros
\usepackage[plainpages=false]{hyperref}
% configure the hyperref package
%http://www.pa.op.dlr.de/~PatrickJoeckel/pdflatex/index.html
\hypersetup{
             %pdftex,    %required for pdflatex
             colorlinks   = true, %Colours links instead of ugly boxes
             urlcolor     = blue, %Colour for external hyperlinks
             linkcolor    = blue, %Colour of internal links
             citecolor   = red, %Colour of citations
             breaklinks=true,   %allow links to break over lines
             %bookmarks=true,    %bookmarks for all entries in the TOC - already defined somewhere
             bookmarksnumbered=true, %put section numbers in bookmarks
             bookmarksopen=true, %open up bookmark three
             debug=true, %extra diagnostic messages on the log file
             % pdf metadata
             pdftitle={ROVIM T2D - Um veiculo autonomo de vigilancia de instalacoes militares},
             pdfauthor={Goncalo Andre},
             pdfsubject={Dissertacao de Mestrado em Engenharia Eletrotecnica e de Computadores}
             %pdfcreator={Document Creator Name}, - pdflatex should fill this
             pdfkeywords={Tese, Dissertacao, MEEC, IST, ROVIM, T2D, Veiculo Eletrico, Robo}
         }

%Use flowcharts
\usepackage{tikz} %tikz will probably only be used in standalone files, but still needs to be declared here
\usetikzlibrary{shapes.geometric, arrows}

%Manage symbols the same way we manage acronyms
%\usepackage[final]{listofsymbols}
\usepackage{wasysym} %latex symbols
\usepackage{symlist}

%typesett Si units with a package
\usepackage{siunitx}
\sisetup{
    output-decimal-marker={,}% just uncomment if you want to use comma as the decimal marker!
}

%to have line breaks after \paragraph{} and subparagraph{}
%\usepackage[raggedright]{titlesec}

%control spacing between items in lists
\usepackage{enumitem}

% Set paragraph counter to alphanumeric mode
\renewcommand{\theparagraph}{\Alph{paragraph}~--}
% increase horizontal space between rows
\renewcommand{\arraystretch}{1.4}

\newcommand{\figref}[1]{Figure \ref{#1}}
\newcommand{\equationref}[1]{Equation (\ref{#1})}
\newcommand{\tableref}[1]{Table (\ref{#1})}

\newcommand{\textreg}{$\textsuperscript{\textregistered}$}

%Specific macros used in this document to replace content, and not configurations:
\newcommand{\swversion}{v1.0}
\newcommand{\hwversion}{v1.0}

%% Page formatting
%page layout learning:
% https://en.wikibooks.org/wiki/LaTeX/Page_Layout
\hoffset 0in    %this is a page layout dimension
\voffset 0in    %this is a page layout dimension

%Alternative set of page geometry
%\oddsidemargin 0.71cm
%\evensidemargin 0.04cm
%\marginparsep 0in
%\topmargin -0.25cm
%\textwidth 15cm
%\textheight 23.5cm

% geometry package provides flexible and easy interface to page dimensions
%define absolute margin values, ignoring individual page layout parameter dimensions
\usepackage[top=2.5cm, bottom=2.5cm, inner=2.9cm, outer=2.5cm]{geometry}

\usepackage{fancyhdr} % costumize page layout
\pagestyle{fancy} %set the fancy style of fancyhdr package (activate the package)
% Some info on latex marks to explain the following commands
%Marks usage: markboth{left head}{right head} \markright{right head}
%
%The \markboth and \markright commands are used in conjunction with the page style myheadings for setting either both or just the right heading. In addition to their use with the myheadings page style, you can use them to override the normal headings in the headings style, since LaTeX uses these same commands to generate those heads. You should note that a left-hand heading is generated by the last \markboth command before the end of the page, while a right-hand heading is generated by the first \markboth or \markright that comes on the page if there is one, otherwise by the last one before the page. 
%The \leftmark contains the Left argument of the Last \markboth on the page, the \rightmark
%contains the Right argument of the fiRst \markboth or the only argument of the fiRst \markright
%on the page. If no marks are present on a page they are “inherited” from the previous page.
%You can influence how chapter, section, and subsection information (only two of them!) is displayed
%by redefining the \chaptermark, \sectionmark, and \subsectionmark commands 4 . You must
%put the redefinition after the first call of \pagestyle{fancy} as this sets up the defaults.
%Let us illustrate this with chapter info. It is made up of three parts:
%• the number (say, 2), displayed by the macro \thechapter
%• the name (in English, Chapter), displayed by the macro \chaptername
%• the title, contained in the argument of \chaptermark.
% So, after redefining, it would look like this:
\renewcommand{\chaptermark}[1]{\markboth{\thechapter.\ #1}{}}
\renewcommand{\sectionmark}[1]{\markright{\thesection\ #1}}
\fancyhf{} %clear the header and footer; other the elements of the default "plain" pagestyle will appear.
%\fancyhead[LE]{\bfseries\nouppercase{\leftmark}}
%\fancyhead[RO]{\bfseries\nouppercase{\rightmark}}
\fancyfoot[LE,RO]{\bfseries\small\thepage} %Set page number (small font, bold [\bfseries]) on Left even page and on Right Odd page 
\renewcommand{\headrulewidth}{0.0pt}    %Set the footer and header line to 0 pt - make it disappear
\renewcommand{\footrulewidth}{0.0pt}
\addtolength{\headheight}{2pt} % make space for the rule
%Previous commands redefine the defaults set by the package fancyhdr.

%define/redefine custom pagestyles
%redefine the plain \pagestyle{}
\fancypagestyle{plain}{% Used in blank pages
   \fancyhead{} % get rid of headers
   \renewcommand{\headrulewidth}{0pt} % and the line
   \renewcommand{\footrulewidth}{0pt}
   \fancyfoot[LE,RO]{\bfseries\small\thepage} %set pagination as above
}

%define the begin pagestyle - for the preamble
\fancypagestyle{begin}{%
   \fancyhead{}
   \renewcommand{\headrulewidth}{0pt}
   \renewcommand{\footrulewidth}{0pt}
   \fancyfoot[LE,RO]{\bfseries\small\thepage}
}
%define the document pagestyle - for the main body of the document
%the lonely comment symbol (%) on the next line is to prevent unwanted whitespace
\fancypagestyle{document}{%Fancy head and foot with lines
	\fancyhf{} 
    %Set the header of Left Even and Rigt Odd margins to bold and make sure it is not spelled as uppercase (See fsncyhdr.pdf, section 9) - usefull for bibliography, for example.
	\fancyhead[LE]{\bfseries\nouppercase{\leftmark}}
	\fancyhead[RO]{\bfseries\nouppercase{\rightmark}}
	\fancyfoot[LE,RO]{\bfseries\small\thepage}
	%\renewcommand{\headrulewidth}{0pt}
	%\renewcommand{\footrulewidth}{0pt}
	\addtolength{\headheight}{2pt} % make space for the rule
}
%define the documentsimple pagestyle
\fancypagestyle{documentsimple}{% Without the fancy head and foot
	\fancyhf{}
	\fancyfoot[LE,RO]{\bfseries\small\thepage}
	%\renewcommand{\headrulewidth}{0pt}
	%\renewcommand{\footrulewidth}{0pt}
	\addtolength{\headheight}{2pt} % make space for the rule
}
%limit the section numbering depth to 3 levels - do not number from paragraph bellow
\setcounter{secnumdepth} {3}
%limit the toc depth to 5 levels
\setcounter{tocdepth} {5}
%Set a subsubsection number to include also the subsection number.
% ex. subsection.subsubsection
\renewcommand{\thesubsubsection}{\thesubsection.\Alph{subsubsection}}

%have line breaks after paragraphs
%\titleformat{\paragraph}[hang]{\normalfont\normalsize\bfseries}{\theparagraph}{1em}{}
%\titlespacing*{\paragraph}{0pt}{3.25ex plus 1ex minus .2ex}{0.5em}

%{subfigure} package is now obsolete and was replace in this document by
%the {subcaption} package. This change removed the following 4 commands.
%% Length from the top of the subfigure box to the begining of the FIGURE
%% box.  Also from the bottom of the CAPTION to the bottom of the subfigure.
%\renewcommand{\subfigtopskip}{0.3 cm}
%space between the subcaption and the edge of the box
%\renewcommand{\subfigbottomskip}{0.2 cm}
%free space between the figure and the caption, if it exists
%\renewcommand{\subfigcapskip}{0.3 cm}
%indentation of the subfigure caption, in relation to the subfigure margins
%\renewcommand{\subfigcapmargin}{0.2 cm}

\graphicspath{{recursos/}}


% My environments
%Automatically numerate and label table rows.
\newcounter{rowno}  %counter for the numbering
\newenvironment{row_labeled_longtable}[2] %two mandatory arguments - table name and disposition, as in the longtable environment
{
\begin{flushleft}   %align table to the left, because it is going to be wide
    %refstepcounter steps the counter and uses it for the following reference
    %start a longtable environment, supress left and right borders, and on the first column put "C(counter)" and the label (wich depends on the counter). Then format the columns according to the argument passed.
    %Rows can only be appended to the table to maintain compatibility
    \begin{longtable}{ @{} >{\refstepcounter{rowno}C\therowno\label{#1:C\therowno}} #2 @{} }
}
{
\end{longtable}
\end{flushleft}
}

%Since we are using a thesis template, with chapters that are not used in this document, We have to modify the numeration
\renewcommand\thechapter{}
\renewcommand\thesection{\arabic{section}}

%-----------------------------------------------------------
%-----------------------------------------------------------
\begin{document}
\pagestyle{begin}
\setcounter{page}{1} \pagenumbering{Alph}
% Add PDF bookmark 
\pdfbookmark[0]{Capa}{Title}

\thispagestyle{empty}
%\begin{flushleft} ~\\ \vspace{-12mm} \hspace{-12mm}  \includegraphics[width=50mm]{recursos/imagens/istlogo} 
\begin{flushleft} ~\\ \vspace{-12mm} \hspace{-12mm}

\begin{figure}[h]
    \begin{subfigure}{0.5\textwidth}
    \includegraphics[width=0.9\linewidth, width=50mm]{recursos/imagens/istlogo} 
    \end{subfigure}
    \begin{subfigure}{0.5\textwidth}
    \includegraphics[width=0.9\linewidth, width=50mm, right]{recursos/imagens/am_logo.jpg}
    \end{subfigure}
\end{figure}

\vspace{50mm} % gr�ficos
\todo{foto com baterias e sem carro}
 \begin{center} \includegraphics[height=50mm]{recursos/imagens/vista_lado_sem_baterias.jpg}  \end{center} % gr�ficos
 \vspace{5mm}
\centering
\LARGE \textbf{ROVIM T2D}
\\ \vspace{10mm}
\Large Manual do utilizador
\\ \vspace{15mm}
%\Large \textbf{Full Name} \\
\vspace{12mm}
%\large Thesis to obtain the Master of Science Degree in
\vspace{2mm}
%\LARGE \textbf{Biomedical Engineering}
\vspace{10mm}
%\large Supervisor(s): Prof./Dr. Lorem Ipsum
\vspace{15mm}
%\Large \textbf{Examinatiom Committee}
\vspace{5mm}
%\large Chairperson:	Prof. Lorem \\
%\large Supervisor: Prof. Lorem Ipsum\\
%\large Co-Supervisor: Prof. Lorem Ipsum \\
%\large Members of the Committe: Dr. Lorem Ipsum \\
%Prof. Lorem Ipsum
% 
\vspace{15mm}
%
\Large \textbf{\todaythesis\today} \\
%\Large \textbf{January 2015} \\
\Large \textbf{\manualversion} \\

\let\thepage\relax
\end{flushleft}
\pagebreak


\clearpage
% Since I am using double sided pages, the second page should be white.
% Remember that when delivering the dissertation, IST requires for the cover to appear twice.

\thispagestyle{empty}
\cleardoublepage

\setcounter{page}{1} \pagenumbering{arabic}
\setcounter{figure}{0}

\baselineskip 18pt % line spacing: -12pt for single spacing
                   %               -18pt for 1 1/2 spacing
                   %               -24pt for double spacingnts}

\input{recursos/listas.tex}
\acresetall     %faz reset as extensoes de acronimos
% %%%%%%%%%%%%%%%%%%%%%%%%%%%%%%%%%%%%%%%%%%%%%%%%%%%%%%%%%%%%%%%%%%%%%%
 % List of acronyms
\pdfbookmark[1]{Lista de Siglas e Acr�nimos}{loac}

\chapter*{Siglas e Acr�nimos}
\label{loac}


% See more at http://staff.science.uva.nl/~polko/HOWTO/LATEX/acronym.html

\todo{Como escrever este acronimo}
\begin{acronym}
    \acro{ROVIM}{ROb� de Vigil�ncia de Instala��es Militares}
    \acro{T2D}{Tra��o, Travagem e Dire��o}
    \acro{SeN}{Sensores e Navega��o}
    \acro{CPC}{Comunica��es e Posto de Controlo}
    \acro{I2C}[I\textsuperscript{2}C]{\emph{Inter-Integrated Circuit}}
    \acro{OSI}{\emph{Open Systems Interconnection}}
    \acro{MDF}{\emph{Medium-Density Fibreboard}, fibra de madeira de m�dia densidade}
    \acro{DC}{\emph{Direct Current}, (corrente cont�nua)}
    \acro{rpm}{rota��es por minuto}
    \acro{NTC}{\emph{Negative Temperature Coefficient}, coeficiente negativo de temperatura}
    \acro{NiMH}{\emph{Nickel Metal Hydride}, n�quel-hidreto metal, uma tecnologia de baterias}
    \acro{LED}{\emph{Light Emitting Diode}, d�odo emissor de luz}
\end{acronym}

\clearpage
\thispagestyle{empty}
\cleardoublepage





\section{Hist�rico de vers�es}
\label{section:historico}
\input{recursos/historico_sw.tex}
\input{recursos/historico_hw.tex}

%%%%%%%%%%%%%%%%%%%%%%%%%%%%%%%%%%%%%%%%%%%%%%%%%%%%%%%%%%%%%%%%%%%%%%%
% List of symbols
\pdfbookmark[1]{Lista de S�mbolos}{los}

\listofsymbols

\clearpage
\thispagestyle{empty}

\cleardoublepage
\baselineskip 18pt

%\pagestyle{document}%Fancy head and foot with lines
\pagestyle{documentsimple}      %Simple head
% %%%%%%%%%%%%%%%%%%%%%%%%%%%%%%%%%%%%%%%%%%%%%%%%%%%%%%%%%%%%%%%%%%%%%%
% Dummy Chapter:
% %%%%%%%%%%%%%%%%%%%%%%%%%%%%%%%%%%%%%%%%%%%%%%%%%%%%%%%%%%%%%%%%%%%%%%

% %%%%%%%%%%%%%%%%%%%%%%%%%%%%%%%%%%%%%%%%%%%%%%%%%%%%%%%%%%%%%%%%%%%%%%
% 
% %%%%%%%%%%%%%%%%%%%%%%%%%%%%%%%%%%%%%%%%%%%%%%%%%%%%%%%%%%%%%%%%%%%%%%

% Corpo.tex - o Conteudo principal do documento (excliu bibliografia e anexos)

\section{Introdu\c{c}\~ao}
\label{sec:int}
Bem-vindo ao manual do utilizador do m�dulo T2D do ROVIM.
Este documento destina-se a todos os utilizadores do ROVIM, e pretende servir de manual de aprendizagem r�pida e guia de campo deste m�dulo do ROVIM.
� recomendada a leitura completa deste manual antes de operar o rob� pela primeira vez ou para esclarecimento de d�vidas durante a sua utiliza��o.
Na primeira parte deste manual � apresentada uma vis�o geral resumida do m�dulo T2D do ROVIM, seguida de instru��es detalhadas sobre as opera��es de manuten��o, transporte e utiliza��o do ve�culo. Por fim apresenta-se um guia sobre
modifica��es e reconfigura��es do sistema.
\section{Considera\c{c}\~oes de seguran\c{c}a}
O ROVIM foi projectado com um �nfase na seguran�a dos utilizadores. A�nda ssim, � um prot�tipo insuficientemente refinado e testado para poder ser usado em condi��es desfavor�veis, ou por utilizadores inexperientes ou impreparados. Esta fragilidade aliada ao peso e pot�ncia do rob�t tornam a sua utiliza��o pot�ncialmente perigosa. Este cap�tulo imp�e aos utilizadores normas que devem ser seguidas constantemente e impreter�velmente para min�mizar os riscos e a gravidade de pot�nciais acidentes.
\begin{comment}
    rascunho das normas:
    -operar em terreno pouco acidentado
    -operar sem chuva
    -manter a frente do ve�culo sempre desimpedida de obst�culos
    -projectar a trajectoria pretendida para o ve�culo antes de o ligar, e garantir que esta se encontra desimpedida de pessoas ou outros obst�culos
    -alertar as pessoas na vizinha�a do ve�culo para o facto de este estar a ser utilizado e os seus perigos
    -l�r e compreender este manual antes de operar o ve�culo pela primeira vez
    -levantar as rodas de tr�s do ch�o quando o desligar
    -aprender a operar o veiculo com as 4 rodas no ar, antes de passar para o ch�o. Familiazirar-se com todas as funcionalidades e procedimentos de seguran�a
    -Manter sempre o dispositivo do homem-morto pronto a disparar (de prefer�ncia atado ao bra�o do utilizador)
\end{comment}
\section{Objer ajuda}
O opera��o do m�dulo T2D do ROVIM pode suscitar d�vidas para os utilizadores mais inexperi�ntes. Existem no entanto v�rias formas de obter ajuda e esclarecer d�vidas:
\begin{comment}
    rascunho do m�todo de obten��o de ajuda
    -consulta do pessoal envolvido no projecto. As pessoas que trabalharam pr�viamente no projecto podem ajudar a esclarecer d�vidas ou dar conselhos sobre a utiliza��o do rob�t. Consulte a lista de colaboradores do projeto ROVIM, e do m�dulo \ref{sec:colab}{T2D} para mais detalhes.
    -consulta das teses e manuais dos v�rios componentes. Cada m�dulo do ROVIM originou uma tese de mestrado que pode ser consultada. Para o m�dulo T2D, a consulta dos [datasheets] dos v�rios componentes pode ser muito �til.
    -consulta do codigo: O \ref{}{c�digo} do controlador do m�dulo T2D pode ser �til no esclarecimento de d�vidas.
\end{comment}
\section{Descri\c{c}\~ao do sistema}
\subsection{Interface do utilizador}
O m�dulo T2D possui v�rios pontos de contacto com o utilizador.
\subsubsection{Acionadores mec�nicos}
\begin{comment}
tabela com os v�rios bot�es, fun��o, estados poss�veis, condicionantes, obs, foto c/identifica��o
\end{comment}
\subsubsection{Linha de comandos}
\subsubsection{I2C}
A vers�o original do software do controlador Dalf possui uma interface por I2C de utiliza��o identica � interface por porta s�rie. No entanto, a interface s�rie foi modificada para esta aplica��o de modo a poder tamb�m mostrar informa��o assincronamente. Esta funcionalidade n�o est� dispon�vel nesta
\subsubsection{R/C}
\subsubsection{Mostradores}
\begin{comment}
tabela com os v�rios leds, fun��o, estados poss�veis, condicionantes, obs, foto c/identifica��o
\end{comment}
\begin{comment}
    tabela com a descri�o da informa��o da porta s�rie, fun��o, estados poss�veis, condicionantes, obs, foto c/identifica��o
\end{comment}
O controlador de tra�ao Signam Drive possui um mostrador de informa��es de estado e diagn�stico [que � remov�vel - mostrar foto], que pode ser usado. Consultar o manual do controlador para mais informa��es [que se��es a consultar e que info se pode obter]
\subsection{Tra\c{c}\~ao}
\subsection{Travagem}
\subsection{Dire\c{c}\~ao}
\subsection{Energia}
\subsection{Seguran\c{c}a}
\subsection{Controlo e comunia\c{c}\~ao}
\section{Manuten\c{c}\~ao e transporte}
\section{Funcionalidades}
\subsection{Modos de opera\c{c}\~ao}
\paragraph{Manual}
\paragraph{Auto - porta s\'erie}
\paragraph{Auto - I2C}



\section{Obter ajuda}
\label{sec:ajuda}
A utiliza��o do \ac{ROVIM} pode suscitar d�vidas a alguns utilizadores mais inexperientes. Existem no entanto v�rias formas de obter ajuda e esclarecer d�vidas ao seu dispor:
\begin{itemize}
    \item \textbf{Este manual.} Aqui s�o esclarecidas a maior parte das d�vidas que podem surgir na utiliza��o do ve�culo. Cont�m informa��o mais atualizada que as teses de mestrado.
    \item \textbf{Documenta��o produzida pelos anteriores colaboradores.} Esta deve conter detalhes sobre o c�digo fonte, esquemas el�tricos e \emph{layout} dos componentes eletr�nicos, entre outra informa��o. Esta documenta��o � de livre acesso, mas pode ser obtida junto dos colaboradores do projeto.
    \item \textbf{Documenta��o dos v�rios componentes individuais.} Muita desta documenta��o est� livremente acess�vel na internet, mas pode ser obtida junto dos colaboradores do projeto. A lista de componentes est� dispon�vel no \nameref{ap:c}.
    \item \textbf{Reposit�rio do c�digo do projeto,} em \url{https://github.com/ROVIM-T2D/ROVIM-T2D-Brain.git}. O c�digo aqui encontrado pode ser mais recente que o c�digo programado no ve�culo.
    \item \textbf{Os \nameref{sec:colaboradores} do projecto}. Os antigos e atuais colaboradores estar�o dispon�veis para ajudar a esclarecer d�vidas e aconselhar os atuais colaboradores e utilizadores.
\end{itemize}


\section{Recomenda��es de seguran\c{c}a}
\label{sec:seguranca}
\subsection{Utiliza��o do aparelho}
O \ac{ROVIM} foi projectado com grande �nfase na preven��o de acidentes. A�nda assim, � um prot�tipo insuficientemente testado e refinado para poder ser usado em condi��es desfavor�veis, ou por utilizadores inexperientes ou impreparados. Esta fragilidade aliada ao peso e pot�ncia do rob� tornam a sua utiliza��o pot�ncialmente perigosa. Este cap�tulo imp�e aos utilizadores normas que devem ser seguidas sempre e sem reserva, para min�mizar os riscos e a gravidade de pot�nciais acidentes.\\
Assim, antes de operar o rob�, os utilizadores devem:
\begin{itemize}
    \item \textbf{Estar familiarizados com este manual;}
    \item \textbf{Activar corretamente o dispositivo do homem-morto.} S� assim � poss�vel parar o ve�culo em seguran�a no caso de perda de controlo. Consulte as sec��es \ref{ssec:descricao_hardware} e \ref{sec:funcionalidades} para mais detalhes;
    \item \textbf{Alertar as pessoas na vizinhan�a da trajet�ria esperada para a presen�a do ve�culo.}
\end{itemize}
Recomenda-se tamb�m que os utilizadores, para al�m da leitura pr�via deste manual, aprendam a operar o rob� com um supervisor antes de passarem a oper�-lo sozinhos.\\
Durante a utiliza��o, o rob� n�o deve:
\begin{itemize}
    \item \textbf{Ter obst�culos � sua frente num raio de 10 metros.} Em caso de perda de controlo, a dist�ncia de seguran�a dar� tempo de rea��o �s pessoas e tempo para o dispositivo do homem-morto atuar;
    \item \textbf{Operar em terreno acidentado e/ou inclinado,} devido ao risco de alguns componentes se deslocarem, podendo provocar curto-circuitos;
    \item \textbf{Apanhar chuva ou �gua,} devido ao risco de curto-circuito em alguma liga��o n�o protegida.
\end{itemize}
No fim de operar o rob�, os utilizadores devem sempre deslig�-lo e armazen�-lo correctamente. Consulte as sec��es \ref{sec:funcionalidades} e \ref{sec:armazenamento} para mais detalhes.\\
\subsection{Manuseamento do material el�trico}
O material el�trico do \ac{ROVIM}, embora precariamente acondicionado, est� suficientemente protegido para prevenir acidentes durante a condu��o normal do rob�. No entanto, as atividades de manuten��o exigem precau��es adicionais. � por isso recomendada a leitura das instru��es das se��es \ref{sec:manutencao}, \ref{sec:transporte}, \ref{sec:armazenamento}.

chapter{Componentes do sistema}
\label{ch:componentes_sistema}
\subsection{\emph{Hardware}}
\label{ssec:descricao_hardware}

\subsubsection{\emph{Chassis}}
O \emph{chassis} � o esqueleto do \ac{ROVIM}. � nele que todos os componentes embarcados do m�dulo \ac{T2D} s�o instalados.

    � uma adapta��o do \emph{chassis} de uma moto-quatro com motor de combust�o interna, de onde foram removidas todas as pe�as n�o essenciais (ver componente \ref{componentes:C1} da lista do ap�ndice \ref{ap:d}). A figura \ref{fig:chassis_despido} mostra o \emph{chassis} antes das adapta��es para esta aplica��o.

    \begin{figure}[h] %place around here
        \centering
        \includegraphics[width=0.7\linewidth]{recursos/imagens/chassi_despido.jpg}
        \caption[\emph{chassis}]{Chassis de moto-quatro antes das adapta��es para o \ac{ROVIM}.}
        \label{fig:chassis_despido}\par %end \centering
    \end{figure}
    Ao chassis foram soldadas as estruturas para fixa��o dos componentes: uma estrutura central de dois n�veis de plataformas, uma estrutura para o. principais, �s quais estes foram aparafusados. Estas s�o: uma estrutura central de plataformas, uma estrutura de fixa��o do motor de tra��o
    Foi constru�da na ba�a do antigo motor de combust�o uma estrutura para duas plataformas de madeira \ac{MDF} empilhadas, para acomodar as baterias (que s�o o componente mais pesado do ve�culo) e n�o alterar significativamente o centro de gravidade projectado pelo construtor da moto. Na plataforma inferior foram instalados batentes para impedir as baterias de deslizar e ganchos para as poder segurar com el�sticos. enquanto que a superior serviu para instalar outros componentes e liga��es.\todo{Mostar fotos da estrutura, plataformas e ba�a das baterias em anexo}

    %estas plataformas sao de \ac{MDF}, madeira leve, resistente e n�o condutora. Apesar disso, o aro e os parafusos usados na plataforma s�o de ferro condutor, devido a limita��es nos materiais dispon�veis. Isso representa um risco adicional de curto-circuito nas baterias que deve ser tomado em conta durante a opera��o da moto. A solu��o desenhada para fixar as baterias no sitio consiste nos batentes de madeira colados e nos el�sticos. � uma solu��o que serve para a prova do conceito do rob�, mas que � fr�gil e impede a condu��o em terrenos acidentados ou inclinados. Est� fora dos requisitos para esta fase prote��es para a chuva.
    Na forquilha traseira foram criados apoios para o conjunto motor de tra��o + redutor, de modo ao carreto de sa�da do redutor assentar no mesmo plano que o carreto do eixo traseiro, assim como um esticador para a corrente.\todo{mostrar foto dos apoios do motor em anexo}. Foi tamb�m criado um suporte para fixar um sensor junto ao carreto de sa�da do redutor, projetado para maximizar as op��es de fixa��o.
    %devido � curta dimensao da corrente, teve que se colocar um esticador, porque o passo dos elos tornava-a muito curta ou muito larga.
    %ficou n�o suspenso - sujeito a mais vibra��es mas elimina o problema de varia��o da distancia entre carretos (acontecia caso fixasse o motor na mesma zona, mas � parte suspensa), libertando assim espa�o para mais baterias na zona central. As vibra��es n�o s�o problema para as condi��es projetadas de uso.
    Foi criada uma pequena plataforma para o motor do trav�o na forquilha traseira, assim como bra�adeiras semi-rigidas para o fixar, e apoios de borracha, ambos desenhados para acomodar a varia��o do �ngulo entre o eixo do piv� do trav�o e o veio deste motor que se d� durante o curso da travagem \todo{est� percet�vel o problema?}.\todo{mostrar foto do piv� do trav�o em anexo, assim como da plataforma e bra�adeiras. Mostrar que o pivo fica mais baixo quando se trava} O trav�o dianteiro foi mantido inalterado, uma vez que n�o interferia com as modifica��es necess�rias e permite travar manualmente a moto.

    Foi criado um apoio para fixar o redutor do motor de dire��o na frente do \emph{chassis} e a coluna da dire��o foi cortada para lhe acoplar uma roda dentada conc�ntrica, situada no mesmo plano que a roda dentada acoplada na sa�da do redutor da dire��o. Infelizmente criou-se neste processo uma zona morta na uni�o das duas partes da coluna: a parte superior da dire��o (acoplada ao guiador) tem que girar um determinado �ngulo\todo{quanto?} at� que a zona inferior (acoplada �s rodas) comece a girar solidariamente.\todo{mostar em anexo foto do corte na coluna da dire��o}
    O resultado final a que de ora se chama \emph{chassis} do \ac{ROVIM} � composto pelos componentes \ref{componentes:C1}, \ref{componentes:C2}, \ref{componentes:C3} -- \ref{componentes:C9}.

    \subsubsection{Baterias}
    Existem 7 baterias \texttt{EnerSys Genesis NP55-12R}; 6 delas montadas em s�rie na moto, e uma bateria suplente. A figura \ref{fig:baterias_cima_frente}, mostra a vista explodida da montagem das baterias fora da moto.

    \begin{figure}[h] %place around here
        \centering %preferable to \begin{center}, because it doesn't add vspace
        \includegraphics[width=0.7\linewidth]{recursos/imagens/baterias_cima_frente.jpg}
        \caption[Baterias]{Vista explodida da montagem das baterias fora da moto. Legenda:\\1: terminal V-; 2: terminal V+; 3: espa�adores; 4: cabo connector.}
        \label{fig:baterias_cima_frente}\par %end \centering
    \end{figure}
    Este \cite{NP55-12R_datasheet} � um modelo de baterias de �cido e chumbo, de 12 V nominais, recarreg�veis e seladas. Cada bateria pesa 17.7 Kg. Nova, tem uma corrente nominal de descarga de 150 A, conseguindo atingir picos de 500 A. A um ritmo constante de descarga de 10 A, a carga de uma bateria dura cerca de 5 horas.

    As baterias s�o ligadas entre si como indicado a figura \ref{fig:baterias_cima_frente} por cabos de elevado calibre\todo{quanto?}, e separadas por espa�adores de madeira, para haver circula��o de ar no interior da montagem. \todo{referir que n�o tive acesso a recursos para fixa��o de baterias em seguran�a, e que seria dificil fixa-las seguramente ao chassis e ainda conseguir remove-las. Assim desenhei esa solu�ao de compromisso}As baterias fixam na plataforma inferior, de lado e atr�s com recurso a batentes colados estrategicamente, suficientemente pequenos para permitir a remo��o das baterias e � frente encostando a duas t�buas de madeira (dispostas a toda a altura das baterias). Cabos el�sticos permitem segurar a parte superior das baterias, ainda que sem grande consist�ncia.\todo{referir anexo com a figura da baia das baterias e foto das baterias montadas}

    A tens�o aos terminais das baterias varia entre os 57,6 V, quando as baterias est�o completamente descarregadas e os 81 V, quando as baterias est�o novas e completamente carregadas.

    Atualmente, as baterias atuais est�o no seu fim de vida (5 anos em repouso \cite{NP55-12R_datasheet}), pelo que devem ser substitu�das.
    \todo{baterias de chumbo-�cido s�o uma tecnologia testada e  fi�vel, mais barata que as � base de l�tio e s�o f�ceis de recarregar.}
    \subsection{Carregadores das baterias}
    Foram adquiridos tamb�m dois carregadores para as baterias do \ac{ROVIM}: um individual de 12 V e outro de 72 V, de modo a poder carregar o conjunto das baterias sem as retirar do ve�culo.
    \subsection{Tra��o}
    O ve�culo � movido por um motor \texttt{Agni B95R}. Este � um motor \cite{Agni_B95R_performance}, \cite{Agni_workshop} \ac{DC} com escovas. A vers�o refor�ada usada nesta aplica��o atinge as 6000 \ac{rpm} e tem uma pot�ncia nominal de 16 KW a 72 V, cerca de 230 A de corrente nominal. A pot�ncia de pico � de cerca de 30 KW, e a corrente de pico cerca de 400 A.

    O motor vem equipado com um term�stor \cite{NTC_datasheet} \ac{NTC}, que permite limitar a corrente no motor quando este aquece demasiado. O ap�ndice \ref{ap:d} cont�m o esquema el�trico da montagem do term�stor. Este term�stor tem um resist�ncia de \todo{quantos?} K$\Omega$ (medida com um mult�metro) � temperatura ambiente, e de cerca de 2 K$\Omega$ � temperatura m�xima aceit�vel para o funcionamento das escovas do motor.

    Este est� acoplado a um redutor projetado e constru�do de raiz para esta aplica��o\todo{dizer pq se f�z um de raiz - que se aproveitaram os carretos e corrente originais da moto, que as op��es eram muito caras, com pouca redu��o, ou transmiss�o de bin�rio unidirecional para as caixas sem-fim coroa-para al�m disso complicava o acoplamento do redutor ao eixo.}. O bin�rio � transmitido do motor ao eixo traseiro por tr�s acoplamentos de engrenagens e carretos, totalizando uma desmultiplica��o total de 19.(63) vezes\todo{confirmar}.

    A constru��o do redutor consiste em duas chapas de metal que cobrem lateralmente as engrenagens e onde assentam os rolamentos do veios, e quatro espa�adores nos cantos das placas, que as fixam em paralelo e com elas formam uma "caixa". O veio do motor fica instalado a uma engrenagem dentro da caixa. � sa�da, um carreto instalado no �ltimo veio (que se propaga para al�m do espa�o entre as duas chapas) transmite o bin�rio para o carreto acoplado ao veio do motor, usando uma corrente.

    A figura \ref{fig:motor_redutor_instalados_3_4}\todo{adicionar figura que mostre as engrenagens da caixa, o motor, e a corrente e os carretos - plano 3/4 sugere-se} mostra a instala��o no ve�culo do conjunto motor + redutor. O ap�ndice \ref{ap:e}\todo{criar ap�ndice D} cont�m os desenhos t�cnicos das pe�as.

O motor � controlado por um \texttt{Sigmadrive PMT835M}. Este � um controlador de quatro quadrantes para motores \ac{DC} de �manes permanentes, de 80 V e 350 A de capacidade nominal \cite{datasheet_PMT835M}, especialmente desenhado para uso em ve�culos el�tricos. O controlador trabalha em conjunto com um sensor de velocidade, de relut�ncia magn�tica vari�vel, constru�do de prop�sito para esta aplica��o e instalado no carreto de sa�da do redutor\todo{figura}. O sensor est� protegido por uma c�psula selada com cola (N�o � poss�vel aceder ao seu interior). A c�psula cont�m um \ac{LED} que pisca de cada vez que o sensor conta um impulso (passa junto a um dente a uma velocidade suficientemente elevada).\todo{Dizer como foi feito e instalado o sensor de velocidade, e as suas caracteristicas, nomeadamente a velocidade minima de funcionamento}.
\subsubsection{Travagem}
A travagem � assegurada por dois sistemas complementares: um el�trico e aut�nomo, e outro manual.
O sistema de travagem manual consiste no trav�o por cabo das duas rodas frontais. Este trav�o n�o foi alterado na convers�o do \emph{chassis} e � atuado por uma manete no guiador. S� pode ser usado por um operador humano com acesso ao guiador.
O sistema el�trico � movido por um servomotor texttt{PARVEX RS430H}. Este � um motor \ac{DC} compacto de �manes permanentes, de 78 V, que atinge as 3000 \ac{rpm} e os 1.36 N.m usado em pequenas aplica��es rob�ticas \cite{datasheet_RS430H}.


%\subsubsection{Energia}
%\label{sssec:energia}
%Fazem parte do subsistema de energia
%\subsubsection{Tra��o}
%\label{sssec:tracao}
%\subsubsection{Travagem}
%\label{sssec:travagem}
%\subsubsection{Dire��o}
%\label{sssec:direcao}
%\subsubsection{Controlo}
%\label{sssec:controlo}

\subsection{\emph{Software}}
\label{ssec:descricao_software}
Nota sobre o sw do sigmad que nao se considera aqui.

\label{versao_sw_instalada} v1.0
\subsubsection{Arquitetura}
\label{sssec:arquitetura}


\section{Interface com o utilizador}
\label{sec:interface}
%analisar melhor como fazer aqui com os bot�es/controlos. � que os controlos s�o uma constru��o sobre a interface. Os but�es � que constituem a dita.


\section{Estados do sistema}
\label{sec:estados}


\subsection{Modo manual}
\label{ssec:estados_manual}



\subsection{Modo autom�tico}
\label{ssec:estados_auto}




\section{Funcionalidades}
\label{sec:funcionalidades}


\subsection{Modo autom�tico}
\label{ssec:funcionalidades_automatico}
\subsubsection{Interace TE}
\label{sssec:interface_te}
\label{auto:desligar}
\subsubsection{Interface I2C}
\label{sssec:interface_i2c}
N/A por enquanto.
\subsubsection{Interface R/C}
\label{sssec:interface_rc}
N/A



\subsection{Modo manual}
\label{ssec:funcionalidades_manual}
\label{manual:desligar}



\section{Manuten\c{c}\~ao}
\label{sec:manutencao}
Garantir que os parafusos est�o apertados, carregar as baterias periodicamente, apertar a correia da dire��o, garantir que os fds n�o est�o tortos.\\
dedicar especial importancia ao trav�o, por causa da sua import�ncia na seguran�a da moto

\section{Transporte}
\label{sec:transporte}
retirar baterias, libertar correia dire��o. retirar cabos soltos.

\section{Armazenamento}
\label{sec:armazenamento}
-retirar chave, fus�vel de pot�ncia, disp. homem-morto; tapar a eletronica; guardar em local pouco frequentado (por causa das baterias), retirar as rodas de tr�s do ch�o.


\section{Resolu��o de problemas}
\label{sec:resolucao_problemas}
-lista erros SigmaD\\
-reinicio dalf ocasionalmente aquando da trv/destrv emerg.\\
-funcionalidades debug\\
-o que fazer num erro desconhecido\\

\section{Problemas conhecidos}
\label{sec:problemas_conhecidos}
-a velocidade mt baixa <~1 km/h a leitura do sensor n�o tem exatid�o.

%\input{conteudo/}
\cleardoublepage


%this is just here temporarily, to have references for the biblio
%% %%%%%%%%%%%%%%%%%%%%%%%%%%%%%%%%%%%%%%%%%%%%%%%%%%%%%%%%%%%%%%%%%%%%%%
% Dummy Chapter:
% %%%%%%%%%%%%%%%%%%%%%%%%%%%%%%%%%%%%%%%%%%%%%%%%%%%%%%%%%%%%%%%%%%%%%%

% %%%%%%%%%%%%%%%%%%%%%%%%%%%%%%%%%%%%%%%%%%%%%%%%%%%%%%%%%%%%%%%%%%%%%%
% 
% %%%%%%%%%%%%%%%%%%%%%%%%%%%%%%%%%%%%%%%%%%%%%%%%%%%%%%%%%%%%%%%%%%%%%%

% Corpo.tex - o Conteudo principal do documento (excliu bibliografia e anexos)

\section{Introdu\c{c}\~ao}
\label{sec:int}
Bem-vindo ao manual do utilizador do m�dulo T2D do ROVIM.
Este documento destina-se a todos os utilizadores do ROVIM, e pretende servir de manual de aprendizagem r�pida e guia de campo deste m�dulo do ROVIM.
� recomendada a leitura completa deste manual antes de operar o rob� pela primeira vez ou para esclarecimento de d�vidas durante a sua utiliza��o.
Na primeira parte deste manual � apresentada uma vis�o geral resumida do m�dulo T2D do ROVIM, seguida de instru��es detalhadas sobre as opera��es de manuten��o, transporte e utiliza��o do ve�culo. Por fim apresenta-se um guia sobre
modifica��es e reconfigura��es do sistema.
\section{Considera\c{c}\~oes de seguran\c{c}a}
O ROVIM foi projectado com um �nfase na seguran�a dos utilizadores. A�nda ssim, � um prot�tipo insuficientemente refinado e testado para poder ser usado em condi��es desfavor�veis, ou por utilizadores inexperientes ou impreparados. Esta fragilidade aliada ao peso e pot�ncia do rob�t tornam a sua utiliza��o pot�ncialmente perigosa. Este cap�tulo imp�e aos utilizadores normas que devem ser seguidas constantemente e impreter�velmente para min�mizar os riscos e a gravidade de pot�nciais acidentes.
\begin{comment}
    rascunho das normas:
    -operar em terreno pouco acidentado
    -operar sem chuva
    -manter a frente do ve�culo sempre desimpedida de obst�culos
    -projectar a trajectoria pretendida para o ve�culo antes de o ligar, e garantir que esta se encontra desimpedida de pessoas ou outros obst�culos
    -alertar as pessoas na vizinha�a do ve�culo para o facto de este estar a ser utilizado e os seus perigos
    -l�r e compreender este manual antes de operar o ve�culo pela primeira vez
    -levantar as rodas de tr�s do ch�o quando o desligar
    -aprender a operar o veiculo com as 4 rodas no ar, antes de passar para o ch�o. Familiazirar-se com todas as funcionalidades e procedimentos de seguran�a
    -Manter sempre o dispositivo do homem-morto pronto a disparar (de prefer�ncia atado ao bra�o do utilizador)
\end{comment}
\section{Objer ajuda}
O opera��o do m�dulo T2D do ROVIM pode suscitar d�vidas para os utilizadores mais inexperi�ntes. Existem no entanto v�rias formas de obter ajuda e esclarecer d�vidas:
\begin{comment}
    rascunho do m�todo de obten��o de ajuda
    -consulta do pessoal envolvido no projecto. As pessoas que trabalharam pr�viamente no projecto podem ajudar a esclarecer d�vidas ou dar conselhos sobre a utiliza��o do rob�t. Consulte a lista de colaboradores do projeto ROVIM, e do m�dulo \ref{sec:colab}{T2D} para mais detalhes.
    -consulta das teses e manuais dos v�rios componentes. Cada m�dulo do ROVIM originou uma tese de mestrado que pode ser consultada. Para o m�dulo T2D, a consulta dos [datasheets] dos v�rios componentes pode ser muito �til.
    -consulta do codigo: O \ref{}{c�digo} do controlador do m�dulo T2D pode ser �til no esclarecimento de d�vidas.
\end{comment}
\section{Descri\c{c}\~ao do sistema}
\subsection{Interface do utilizador}
O m�dulo T2D possui v�rios pontos de contacto com o utilizador.
\subsubsection{Acionadores mec�nicos}
\begin{comment}
tabela com os v�rios bot�es, fun��o, estados poss�veis, condicionantes, obs, foto c/identifica��o
\end{comment}
\subsubsection{Linha de comandos}
\subsubsection{I2C}
A vers�o original do software do controlador Dalf possui uma interface por I2C de utiliza��o identica � interface por porta s�rie. No entanto, a interface s�rie foi modificada para esta aplica��o de modo a poder tamb�m mostrar informa��o assincronamente. Esta funcionalidade n�o est� dispon�vel nesta
\subsubsection{R/C}
\subsubsection{Mostradores}
\begin{comment}
tabela com os v�rios leds, fun��o, estados poss�veis, condicionantes, obs, foto c/identifica��o
\end{comment}
\begin{comment}
    tabela com a descri�o da informa��o da porta s�rie, fun��o, estados poss�veis, condicionantes, obs, foto c/identifica��o
\end{comment}
O controlador de tra�ao Signam Drive possui um mostrador de informa��es de estado e diagn�stico [que � remov�vel - mostrar foto], que pode ser usado. Consultar o manual do controlador para mais informa��es [que se��es a consultar e que info se pode obter]
\subsection{Tra\c{c}\~ao}
\subsection{Travagem}
\subsection{Dire\c{c}\~ao}
\subsection{Energia}
\subsection{Seguran\c{c}a}
\subsection{Controlo e comunia\c{c}\~ao}
\section{Manuten\c{c}\~ao e transporte}
\section{Funcionalidades}
\subsection{Modos de opera\c{c}\~ao}
\paragraph{Manual}
\paragraph{Auto - porta s\'erie}
\paragraph{Auto - I2C}



\section{Obter ajuda}
\label{sec:ajuda}
A utiliza��o do \ac{ROVIM} pode suscitar d�vidas a alguns utilizadores mais inexperientes. Existem no entanto v�rias formas de obter ajuda e esclarecer d�vidas ao seu dispor:
\begin{itemize}
    \item \textbf{Este manual.} Aqui s�o esclarecidas a maior parte das d�vidas que podem surgir na utiliza��o do ve�culo. Cont�m informa��o mais atualizada que as teses de mestrado.
    \item \textbf{Documenta��o produzida pelos anteriores colaboradores.} Esta deve conter detalhes sobre o c�digo fonte, esquemas el�tricos e \emph{layout} dos componentes eletr�nicos, entre outra informa��o. Esta documenta��o � de livre acesso, mas pode ser obtida junto dos colaboradores do projeto.
    \item \textbf{Documenta��o dos v�rios componentes individuais.} Muita desta documenta��o est� livremente acess�vel na internet, mas pode ser obtida junto dos colaboradores do projeto. A lista de componentes est� dispon�vel no \nameref{ap:c}.
    \item \textbf{Reposit�rio do c�digo do projeto,} em \url{https://github.com/ROVIM-T2D/ROVIM-T2D-Brain.git}. O c�digo aqui encontrado pode ser mais recente que o c�digo programado no ve�culo.
    \item \textbf{Os \nameref{sec:colaboradores} do projecto}. Os antigos e atuais colaboradores estar�o dispon�veis para ajudar a esclarecer d�vidas e aconselhar os atuais colaboradores e utilizadores.
\end{itemize}


\section{Recomenda��es de seguran\c{c}a}
\label{sec:seguranca}
\subsection{Utiliza��o do aparelho}
O \ac{ROVIM} foi projectado com grande �nfase na preven��o de acidentes. A�nda assim, � um prot�tipo insuficientemente testado e refinado para poder ser usado em condi��es desfavor�veis, ou por utilizadores inexperientes ou impreparados. Esta fragilidade aliada ao peso e pot�ncia do rob� tornam a sua utiliza��o pot�ncialmente perigosa. Este cap�tulo imp�e aos utilizadores normas que devem ser seguidas sempre e sem reserva, para min�mizar os riscos e a gravidade de pot�nciais acidentes.\\
Assim, antes de operar o rob�, os utilizadores devem:
\begin{itemize}
    \item \textbf{Estar familiarizados com este manual;}
    \item \textbf{Activar corretamente o dispositivo do homem-morto.} S� assim � poss�vel parar o ve�culo em seguran�a no caso de perda de controlo. Consulte as sec��es \ref{ssec:descricao_hardware} e \ref{sec:funcionalidades} para mais detalhes;
    \item \textbf{Alertar as pessoas na vizinhan�a da trajet�ria esperada para a presen�a do ve�culo.}
\end{itemize}
Recomenda-se tamb�m que os utilizadores, para al�m da leitura pr�via deste manual, aprendam a operar o rob� com um supervisor antes de passarem a oper�-lo sozinhos.\\
Durante a utiliza��o, o rob� n�o deve:
\begin{itemize}
    \item \textbf{Ter obst�culos � sua frente num raio de 10 metros.} Em caso de perda de controlo, a dist�ncia de seguran�a dar� tempo de rea��o �s pessoas e tempo para o dispositivo do homem-morto atuar;
    \item \textbf{Operar em terreno acidentado e/ou inclinado,} devido ao risco de alguns componentes se deslocarem, podendo provocar curto-circuitos;
    \item \textbf{Apanhar chuva ou �gua,} devido ao risco de curto-circuito em alguma liga��o n�o protegida.
\end{itemize}
No fim de operar o rob�, os utilizadores devem sempre deslig�-lo e armazen�-lo correctamente. Consulte as sec��es \ref{sec:funcionalidades} e \ref{sec:armazenamento} para mais detalhes.\\
\subsection{Manuseamento do material el�trico}
O material el�trico do \ac{ROVIM}, embora precariamente acondicionado, est� suficientemente protegido para prevenir acidentes durante a condu��o normal do rob�. No entanto, as atividades de manuten��o exigem precau��es adicionais. � por isso recomendada a leitura das instru��es das se��es \ref{sec:manutencao}, \ref{sec:transporte}, \ref{sec:armazenamento}.

chapter{Componentes do sistema}
\label{ch:componentes_sistema}
\subsection{\emph{Hardware}}
\label{ssec:descricao_hardware}

\subsubsection{\emph{Chassis}}
O \emph{chassis} � o esqueleto do \ac{ROVIM}. � nele que todos os componentes embarcados do m�dulo \ac{T2D} s�o instalados.

    � uma adapta��o do \emph{chassis} de uma moto-quatro com motor de combust�o interna, de onde foram removidas todas as pe�as n�o essenciais (ver componente \ref{componentes:C1} da lista do ap�ndice \ref{ap:d}). A figura \ref{fig:chassis_despido} mostra o \emph{chassis} antes das adapta��es para esta aplica��o.

    \begin{figure}[h] %place around here
        \centering
        \includegraphics[width=0.7\linewidth]{recursos/imagens/chassi_despido.jpg}
        \caption[\emph{chassis}]{Chassis de moto-quatro antes das adapta��es para o \ac{ROVIM}.}
        \label{fig:chassis_despido}\par %end \centering
    \end{figure}
    Ao chassis foram soldadas as estruturas para fixa��o dos componentes: uma estrutura central de dois n�veis de plataformas, uma estrutura para o. principais, �s quais estes foram aparafusados. Estas s�o: uma estrutura central de plataformas, uma estrutura de fixa��o do motor de tra��o
    Foi constru�da na ba�a do antigo motor de combust�o uma estrutura para duas plataformas de madeira \ac{MDF} empilhadas, para acomodar as baterias (que s�o o componente mais pesado do ve�culo) e n�o alterar significativamente o centro de gravidade projectado pelo construtor da moto. Na plataforma inferior foram instalados batentes para impedir as baterias de deslizar e ganchos para as poder segurar com el�sticos. enquanto que a superior serviu para instalar outros componentes e liga��es.\todo{Mostar fotos da estrutura, plataformas e ba�a das baterias em anexo}

    %estas plataformas sao de \ac{MDF}, madeira leve, resistente e n�o condutora. Apesar disso, o aro e os parafusos usados na plataforma s�o de ferro condutor, devido a limita��es nos materiais dispon�veis. Isso representa um risco adicional de curto-circuito nas baterias que deve ser tomado em conta durante a opera��o da moto. A solu��o desenhada para fixar as baterias no sitio consiste nos batentes de madeira colados e nos el�sticos. � uma solu��o que serve para a prova do conceito do rob�, mas que � fr�gil e impede a condu��o em terrenos acidentados ou inclinados. Est� fora dos requisitos para esta fase prote��es para a chuva.
    Na forquilha traseira foram criados apoios para o conjunto motor de tra��o + redutor, de modo ao carreto de sa�da do redutor assentar no mesmo plano que o carreto do eixo traseiro, assim como um esticador para a corrente.\todo{mostrar foto dos apoios do motor em anexo}. Foi tamb�m criado um suporte para fixar um sensor junto ao carreto de sa�da do redutor, projetado para maximizar as op��es de fixa��o.
    %devido � curta dimensao da corrente, teve que se colocar um esticador, porque o passo dos elos tornava-a muito curta ou muito larga.
    %ficou n�o suspenso - sujeito a mais vibra��es mas elimina o problema de varia��o da distancia entre carretos (acontecia caso fixasse o motor na mesma zona, mas � parte suspensa), libertando assim espa�o para mais baterias na zona central. As vibra��es n�o s�o problema para as condi��es projetadas de uso.
    Foi criada uma pequena plataforma para o motor do trav�o na forquilha traseira, assim como bra�adeiras semi-rigidas para o fixar, e apoios de borracha, ambos desenhados para acomodar a varia��o do �ngulo entre o eixo do piv� do trav�o e o veio deste motor que se d� durante o curso da travagem \todo{est� percet�vel o problema?}.\todo{mostrar foto do piv� do trav�o em anexo, assim como da plataforma e bra�adeiras. Mostrar que o pivo fica mais baixo quando se trava} O trav�o dianteiro foi mantido inalterado, uma vez que n�o interferia com as modifica��es necess�rias e permite travar manualmente a moto.

    Foi criado um apoio para fixar o redutor do motor de dire��o na frente do \emph{chassis} e a coluna da dire��o foi cortada para lhe acoplar uma roda dentada conc�ntrica, situada no mesmo plano que a roda dentada acoplada na sa�da do redutor da dire��o. Infelizmente criou-se neste processo uma zona morta na uni�o das duas partes da coluna: a parte superior da dire��o (acoplada ao guiador) tem que girar um determinado �ngulo\todo{quanto?} at� que a zona inferior (acoplada �s rodas) comece a girar solidariamente.\todo{mostar em anexo foto do corte na coluna da dire��o}
    O resultado final a que de ora se chama \emph{chassis} do \ac{ROVIM} � composto pelos componentes \ref{componentes:C1}, \ref{componentes:C2}, \ref{componentes:C3} -- \ref{componentes:C9}.

    \subsubsection{Baterias}
    Existem 7 baterias \texttt{EnerSys Genesis NP55-12R}; 6 delas montadas em s�rie na moto, e uma bateria suplente. A figura \ref{fig:baterias_cima_frente}, mostra a vista explodida da montagem das baterias fora da moto.

    \begin{figure}[h] %place around here
        \centering %preferable to \begin{center}, because it doesn't add vspace
        \includegraphics[width=0.7\linewidth]{recursos/imagens/baterias_cima_frente.jpg}
        \caption[Baterias]{Vista explodida da montagem das baterias fora da moto. Legenda:\\1: terminal V-; 2: terminal V+; 3: espa�adores; 4: cabo connector.}
        \label{fig:baterias_cima_frente}\par %end \centering
    \end{figure}
    Este \cite{NP55-12R_datasheet} � um modelo de baterias de �cido e chumbo, de 12 V nominais, recarreg�veis e seladas. Cada bateria pesa 17.7 Kg. Nova, tem uma corrente nominal de descarga de 150 A, conseguindo atingir picos de 500 A. A um ritmo constante de descarga de 10 A, a carga de uma bateria dura cerca de 5 horas.

    As baterias s�o ligadas entre si como indicado a figura \ref{fig:baterias_cima_frente} por cabos de elevado calibre\todo{quanto?}, e separadas por espa�adores de madeira, para haver circula��o de ar no interior da montagem. \todo{referir que n�o tive acesso a recursos para fixa��o de baterias em seguran�a, e que seria dificil fixa-las seguramente ao chassis e ainda conseguir remove-las. Assim desenhei esa solu�ao de compromisso}As baterias fixam na plataforma inferior, de lado e atr�s com recurso a batentes colados estrategicamente, suficientemente pequenos para permitir a remo��o das baterias e � frente encostando a duas t�buas de madeira (dispostas a toda a altura das baterias). Cabos el�sticos permitem segurar a parte superior das baterias, ainda que sem grande consist�ncia.\todo{referir anexo com a figura da baia das baterias e foto das baterias montadas}

    A tens�o aos terminais das baterias varia entre os 57,6 V, quando as baterias est�o completamente descarregadas e os 81 V, quando as baterias est�o novas e completamente carregadas.

    Atualmente, as baterias atuais est�o no seu fim de vida (5 anos em repouso \cite{NP55-12R_datasheet}), pelo que devem ser substitu�das.
    \todo{baterias de chumbo-�cido s�o uma tecnologia testada e  fi�vel, mais barata que as � base de l�tio e s�o f�ceis de recarregar.}
    \subsection{Carregadores das baterias}
    Foram adquiridos tamb�m dois carregadores para as baterias do \ac{ROVIM}: um individual de 12 V e outro de 72 V, de modo a poder carregar o conjunto das baterias sem as retirar do ve�culo.
    \subsection{Tra��o}
    O ve�culo � movido por um motor \texttt{Agni B95R}. Este � um motor \cite{Agni_B95R_performance}, \cite{Agni_workshop} \ac{DC} com escovas. A vers�o refor�ada usada nesta aplica��o atinge as 6000 \ac{rpm} e tem uma pot�ncia nominal de 16 KW a 72 V, cerca de 230 A de corrente nominal. A pot�ncia de pico � de cerca de 30 KW, e a corrente de pico cerca de 400 A.

    O motor vem equipado com um term�stor \cite{NTC_datasheet} \ac{NTC}, que permite limitar a corrente no motor quando este aquece demasiado. O ap�ndice \ref{ap:d} cont�m o esquema el�trico da montagem do term�stor. Este term�stor tem um resist�ncia de \todo{quantos?} K$\Omega$ (medida com um mult�metro) � temperatura ambiente, e de cerca de 2 K$\Omega$ � temperatura m�xima aceit�vel para o funcionamento das escovas do motor.

    Este est� acoplado a um redutor projetado e constru�do de raiz para esta aplica��o\todo{dizer pq se f�z um de raiz - que se aproveitaram os carretos e corrente originais da moto, que as op��es eram muito caras, com pouca redu��o, ou transmiss�o de bin�rio unidirecional para as caixas sem-fim coroa-para al�m disso complicava o acoplamento do redutor ao eixo.}. O bin�rio � transmitido do motor ao eixo traseiro por tr�s acoplamentos de engrenagens e carretos, totalizando uma desmultiplica��o total de 19.(63) vezes\todo{confirmar}.

    A constru��o do redutor consiste em duas chapas de metal que cobrem lateralmente as engrenagens e onde assentam os rolamentos do veios, e quatro espa�adores nos cantos das placas, que as fixam em paralelo e com elas formam uma "caixa". O veio do motor fica instalado a uma engrenagem dentro da caixa. � sa�da, um carreto instalado no �ltimo veio (que se propaga para al�m do espa�o entre as duas chapas) transmite o bin�rio para o carreto acoplado ao veio do motor, usando uma corrente.

    A figura \ref{fig:motor_redutor_instalados_3_4}\todo{adicionar figura que mostre as engrenagens da caixa, o motor, e a corrente e os carretos - plano 3/4 sugere-se} mostra a instala��o no ve�culo do conjunto motor + redutor. O ap�ndice \ref{ap:e}\todo{criar ap�ndice D} cont�m os desenhos t�cnicos das pe�as.

O motor � controlado por um \texttt{Sigmadrive PMT835M}. Este � um controlador de quatro quadrantes para motores \ac{DC} de �manes permanentes, de 80 V e 350 A de capacidade nominal \cite{datasheet_PMT835M}, especialmente desenhado para uso em ve�culos el�tricos. O controlador trabalha em conjunto com um sensor de velocidade, de relut�ncia magn�tica vari�vel, constru�do de prop�sito para esta aplica��o e instalado no carreto de sa�da do redutor\todo{figura}. O sensor est� protegido por uma c�psula selada com cola (N�o � poss�vel aceder ao seu interior). A c�psula cont�m um \ac{LED} que pisca de cada vez que o sensor conta um impulso (passa junto a um dente a uma velocidade suficientemente elevada).\todo{Dizer como foi feito e instalado o sensor de velocidade, e as suas caracteristicas, nomeadamente a velocidade minima de funcionamento}.
\subsubsection{Travagem}
A travagem � assegurada por dois sistemas complementares: um el�trico e aut�nomo, e outro manual.
O sistema de travagem manual consiste no trav�o por cabo das duas rodas frontais. Este trav�o n�o foi alterado na convers�o do \emph{chassis} e � atuado por uma manete no guiador. S� pode ser usado por um operador humano com acesso ao guiador.
O sistema el�trico � movido por um servomotor texttt{PARVEX RS430H}. Este � um motor \ac{DC} compacto de �manes permanentes, de 78 V, que atinge as 3000 \ac{rpm} e os 1.36 N.m usado em pequenas aplica��es rob�ticas \cite{datasheet_RS430H}.


%\subsubsection{Energia}
%\label{sssec:energia}
%Fazem parte do subsistema de energia
%\subsubsection{Tra��o}
%\label{sssec:tracao}
%\subsubsection{Travagem}
%\label{sssec:travagem}
%\subsubsection{Dire��o}
%\label{sssec:direcao}
%\subsubsection{Controlo}
%\label{sssec:controlo}

\subsection{\emph{Software}}
\label{ssec:descricao_software}
Nota sobre o sw do sigmad que nao se considera aqui.

\label{versao_sw_instalada} v1.0
\subsubsection{Arquitetura}
\label{sssec:arquitetura}


\section{Interface com o utilizador}
\label{sec:interface}
%analisar melhor como fazer aqui com os bot�es/controlos. � que os controlos s�o uma constru��o sobre a interface. Os but�es � que constituem a dita.


\section{Estados do sistema}
\label{sec:estados}


\subsection{Modo manual}
\label{ssec:estados_manual}



\subsection{Modo autom�tico}
\label{ssec:estados_auto}




\section{Funcionalidades}
\label{sec:funcionalidades}


\subsection{Modo autom�tico}
\label{ssec:funcionalidades_automatico}
\subsubsection{Interace TE}
\label{sssec:interface_te}
\label{auto:desligar}
\subsubsection{Interface I2C}
\label{sssec:interface_i2c}
N/A por enquanto.
\subsubsection{Interface R/C}
\label{sssec:interface_rc}
N/A



\subsection{Modo manual}
\label{ssec:funcionalidades_manual}
\label{manual:desligar}



\section{Manuten\c{c}\~ao}
\label{sec:manutencao}
Garantir que os parafusos est�o apertados, carregar as baterias periodicamente, apertar a correia da dire��o, garantir que os fds n�o est�o tortos.\\
dedicar especial importancia ao trav�o, por causa da sua import�ncia na seguran�a da moto

\section{Transporte}
\label{sec:transporte}
retirar baterias, libertar correia dire��o. retirar cabos soltos.

\section{Armazenamento}
\label{sec:armazenamento}
-retirar chave, fus�vel de pot�ncia, disp. homem-morto; tapar a eletronica; guardar em local pouco frequentado (por causa das baterias), retirar as rodas de tr�s do ch�o.


\section{Resolu��o de problemas}
\label{sec:resolucao_problemas}
-lista erros SigmaD\\
-reinicio dalf ocasionalmente aquando da trv/destrv emerg.\\
-funcionalidades debug\\
-o que fazer num erro desconhecido\\

\section{Problemas conhecidos}
\label{sec:problemas_conhecidos}
-a velocidade mt baixa <~1 km/h a leitura do sensor n�o tem exatid�o.

%\input{conteudo/}
\cleardoublepage



\phantomsection     %http://tex.stackexchange.com/questions/44088/when-do-i-need-to-invoke-phantomsection
\addcontentsline{toc}{chapter}{Bibliografia}    %Add bibliography to toc
%http://tex.stackexchange.com/questions/17128/using-bibtex-to-make-a-list-of-references-without-having-citations-in-the-body-of
\nocite{*}      %show bibliography even if there are no citations
\bibliographystyle{IEEEtran}
\bibliography{recursos/biblio}
\todo{colocar bibliografia em portugues}
\cleardoublepage


\begin{appendices}
	\begin{appendix}
		\pagenumbering{bychapter}
		
\fancychapter{Ap�ndice A - c�digo do programa}
\label{ap:a}


\cleardoublepage
   
        
\fancychapter{Ap�ndice B - Esquemas el�tricos e \emph{layout} das placas eletr�nicas}
\label{ap:b}


\cleardoublepage

        
\fancychapter{Ap�ndice C - Lista de componentes}
\label{ap:c}


\cleardoublepage

        \fancychapter{Ap�ndice D - Lista de componentes}
\label{ap:d}

%\begin{table}
%\begin{center}
%\begin{longtable}{ @{} m{0.05\textwidth} m{0.18\textwidth} m{0.6\textwidth} m{0.1\textwidth} @{} }
\newcounter{compno}  %counter for the numbering
\begin{row_labeled_longtable}{componentes}{ m{0.05\textwidth} m{0.05\textwidth} m{0.18\textwidth} m{0.6\textwidth} m{0.1\textwidth}} {C}{compno}{\thecompno}
    \toprule
    %the multicolum supresses the rule for the automatic component count of this environment. See: http://tex.stackexchange.com/questions/120648/changing-the-table-row-height-in-all-rows-except-the-first-row?rq=1
    %Bug: Idk how to supress left border space in the multicolumn - so it disaligns with the rest of column - bummer
    \multicolumn{1}{ l }{Id.\footnote{Identifica��o}} & Qt.\footnote{Quantidade} & Componente & Descri��o & Subsistema\footnote{Pe�a a que o componente pertence, se aplic�vel}\\
    \midrule
    & 1 & \emph{Chassis} & \emph{Chassis}, rodas, eixo traseiro com carreto de transmiss�o e sistemas de travagem e viragem de uma moto-quatro Suzuki quadsport lt-160& \emph{Chassis}\\
    & Indef.\footnote{Quantidade indefinida} & Ferro sortido & Ferro usado nas estruturas de fixa��o de componentes\footnote{Que s�o: 1) Estrutura central de fixa��o das plataformas; 2) estrutura de suporte do redutor da dire��o; 3) estrutura de fixa��o do motor da tra��o; 4) estrutura de fixa��o do sensor de velocidade; 5) estrutura de fixa��o do sensor de posi��o} e outras adapta��es soldadas ao \emph{Chassis} & \emph{Chassis}\\
    %In. & Parafusos & Parafusos de ferro de v�rias medidas, usados na fixa��o dos componentes ao \emph{chassis} & \emph{Chassis}\\
    %In. & Anilhas & Anilhas de ferro de v�rias medidas, usadas na fixa��o dos componentes ao \emph{chassis} & \emph{Chassis}\\
    %In. & Porcas & Porcas de ferro de v�rias medidas, usadas na fixa��o dos componentes ao \emph{chassis} & \emph{Chassis}\\
    & Indef. & Material de fixa��o & Porcas, parafusos, anilhas, anilhas de mola, cavilhas e outros materiais n�o discriminados de diversas medidas, usados na fixa��o r�gida dos componentes, entre si, ou ao \emph{chassis} &\\
    & 1 & Plataforma inferior & Plataforma de madeira \ac{MDF} cortada, furada e escareada, � medida para a estrutura inferior & \emph{Chassis}\\
    & 1 & Mini-plataforma inferior & Plataforma de madeira \ac{MDF} cortada, furada e escareada, entre a estrutura inferior e a coluna da dire��o & \emph{Chassis}\\
    & 2 & T�buas de madeira & T�buas de madeira, cortadas � medida, fixadas verticalmente � parte frontal da estrutura de suporte das plataformas, para conten��o das baterias & \emph{Chassis}\\
    & 1 & Plataforma superior bombordo & Plataforma de madeira \ac{MDF} cortada, furada e escareada, � medida para o lado de bombordo da estrutura superior & \emph{Chassis}\\
    & 1 & Plataforma superior estibordo & Plataforma de madeira \ac{MDF} cortada, furada e escareada, � medida para o lado de estibordo da estrutura superior & \emph{Chassis}\\
    & 7 & Cantoneiras de madeira & Cantoneiras de madeira, de tamanhos diversos, coladas � plataforma inferior, para conten��o das baterias ao n�vel da base& \emph{chassis}\\
    & 6 & Baterias NP55-12R & Baterias recarreg�veis seladas de �cido-chumbo, de 12 V & Baterias\\
    & 4 & El�sticos de reten��o de cargas & El�sticos, de tamanhos diversos, com ganchos de ferro nas pontas, para abra�ar e conter as baterias ao n�vel do topo & \\
    & 1 & Carregador de baterias 12 V & Carregador de baterias de �cido-chumbo de 12 V &\\
    & 1 & Carregador de baterias 72 V & Carregador de baterias de �cido-chumbo de 72 V &\\
    & 1 & Agni B-95R & Motor \ac{DC} com escovas & \\
    & 1 & Carreto de 11 dentes & Carreto compat�vel com o carreto instalado no veio traseiro & Redutor da tra��o\\
    & 2 & Engrenagens 42 dentes & Engrenagens cil�ndricas, de m�dulo 2, �ngulo de press�o de 20�, material \texttt{C 43 UNI 7847}, de acordo com o cat�logo eurocorreias 2012 \cite{catalogo_eurocorreias} & Redutor da tra��o\\
    & 1 & Engrenagem 21 dentes & Engrenagem cil�ndrica, de m�dulo 2, �ngulo de press�o de 20�, material \texttt{C 43 UNI 7847}, de acordo com o cat�logo eurocorreias 2012 \cite{catalogo_eurocorreias} & Redutor da tra��o\\
    & 1 & Engrenagem 14 dentes & Engrenagem cil�ndrica, de m�dulo 2, �ngulo de press�o de 20�, material \texttt{C 43 UNI 7847}, de acordo com o cat�logo eurocorreias 2012 \cite{catalogo_eurocorreias} & Redutor da tra��o\\
    & 2 & Rolamentos \texttt{Koyo} 6001 & Rolamentos ranhurados de esferas, como especifica��o do catalogo \texttt{Koyo} \cite{catalogo_rolamentos} & Redutor da tra��o\\
    & 2 & Rolamentos \texttt{Koyo} 6003 & Rolamentos ranhurados de esferas, como especifica��o do catalogo \texttt{Koyo} \cite{catalogo_rolamentos} & Redutor da tra��o\\
    & 1 & Chave de veio & Chave para fixa��o de engrenagem ao veio do motor de tra��o, de acordo com a norma ISO/R773 \cite{norma_ISO_R773} & Redutor da tra��o\\
    %& 1 & Freio & Freio para segurar veio em posi��o\todo{confirmar} & Redutor da tra��o\\
    & 1 & Espa�ador para veio & Espa�ador para segurar engrenagem no veio do motor & Redutor da tra��o\\
    & 1 & Fixador para carreto & Bolacha com furo para fixar carreto acoplado ao redutor no veio & Redutor da tra��o\\
    & 1 & Chapa motor & Chapa estrutural de fixa��o dos componentes do redutor do motor, de acordo com o desenho "chapa motor" do ap�ndice \ref{ap:d} & Redutor da tra��o\\
    & 1 & Chapa corrente & Chapa estrutural de fixa��o dos componentes do redutor do motor, de acordo com o desenho "chapa corrente" do ap�ndice \ref{ap:d} & Redutor da tra��o\\
    & 1 & Veio 16 mm & Veio de fixa��o de engrenagens do redutor do motor, de acordo com o desenho "veio 16.12" do ap�ndice \ref{ap:d} & Redutor da tra��o\\
    & 1 & Veio 17 mm & Veio de fixa��o de engrenagens do redutor do motor, de acordo com o desenho "20.23" do ap�ndice \ref{ap:d} & Redutor da tra��o\\
    & Indef. & Material de fixa��o do redutor & Freio, parafusos, anilhas e anilhas de mola de diversas medidas usadas na fixa��o dos componentes do redutor do motor de tra��o & Redutor da tra��o\\
    & 1 & Corrente & Corrente original da moto-quatro & \\
    & 1 & \ac{PCB} sensor & Placa eletr�nica, de cerca de \SI{11}{\milli\meter} por \SI{22}{\milli\meter} com o circuito do esquema \texttt{Encoder} impresso (exceto bobina e n�cleo)\todo{indicar figura do anexo F} & Svel.\footnote{Sensor de velocidade}\\
    & 1 & OPA 2705 \ac{SMD} & Amplificador operacional \emph{rail--to--rail} & Svel.\\
    & 2 & C \SI{100}{\pico\farad} \ac{SMD} & Condensadores de pol�mero & Svel.\\
    & 1 & R \SI{2.2}{\mega\ohm} \ac{SMD} & Resist�ncia & Svel.\\
    & 2 & R \SI{820}{\kilo\ohm} \ac{SMD} & Resist�ncias & Svel.\\
    & 1 & R \SI{620}{\ohm} \ac{SMD} & Resist�ncia & Svel.\\
    & 1 & \ac{LED} vermelho \ac{SMD} & \ac{LED} & Svel.\\
    & 1 & D�odo \ac{SMD} & D�odo de retroa��o \todo{tentar saber qual �} \\
    & 4 & Pinos \ac{SMD} & Pinos conetores para soldar em placa \ac{SMD} & Svel.\\
    & 1 & Bobina artesanal & Bobina em fio de cobre de \SI{0.08}{\milli\meter} com \num{800} espiras & Svel.\\
    & 1 & Enrolador bobina & Enrolador oco, de cerca de \SI{10}{\milli\meter} de diametro exterior, onde assentam espiras de uma bobina & Svel.\\
    & 1 & �mane neod�mio & �mane permanente cil�ndrico, de cerca de \SI{10}{\milli\meter} de diametro, para sensor de relut�ncia vari�vel & Svel.\\
    & 1 & N�cleo ferromagn�tico & Material ferromagn�tico cil�ndrico para sensor de relut�ncia vari�vel & Svel.\\
    & 1 & Tubo \ac{PVC} & Tubo de \ac{PVC} com cerca de \SI{5}{\centi\meter} de comprimento, oco, de cerca de \SI{12}{\milli\meter} de diametro interior & Svel.\\
    & 1 & Cabo de fita & Cabo de fita, de 3 fios, com $\approx$ \SI{5}{\centi\meter} de comprimento & Svel.\\
    & 1 & Adaptador sensor velocidade & Adaptador de ferro, feito sob medida, para fixar o sensor de velocidade no suporte & \\
    & 1 & Bra�adeira mangueira & Bra�adeira para mangueira de jardim & \\
    & 1 & Plataforma trav�o & Plataforma em chapa de \SI{2}{\milli\meter} para fixar em cima da forquilha traseira, onde assenta o motor do trav�o &\\
    & 4 & Anilhas borracha & Anilhas com cerca de \SI{3}{\milli\meter} de espessura, para fixar na plataforma do trav�o &\\
    & 2 & Parker RS430H & Servomotores \ac{DC} &\\
    & 1 & Bra�adeira topo motor & Fixador em chapa, feito sob medida, para fixar o topo do motor do trav�o � forquilha e � plataforma &\\
    & 1 & Bra�adeira tr�s motor & Bra�adeira em chapa, feita sob medida, para fixar o corpo do motor do trav�o � forquilha traseira e � plataforma &\\
    & 1 & Adaptador veio--parafuso & Adaptador cil�ndrico em lat�o com rosca conc�ntrica M10 de um topo e furo \diameter \SI{11}{\milli\meter} conc�ntrico do outro, e furo radial \diameter \SI{4}{\milli\meter} alinhado com furo no veio do motor onde prende& \\
    & 1 & Var�o roscado M10 & Parafuso sem fim do trav�o, com cerca de \SI{40}{\centi\meter} de comprimento &\\
    & 1 & Piv� trav�o & Pe�a, feita sob medida, com rosca longitudinal M10, com duas sali�ncias laterais, para prender ao parafuso sem fim do trav�o &\\
    & 1 & Fixador trav�o--piv� & Conjunto agregador do piv� e alavanca do trav�o, const�tuido por: parafuso e porca M12x\SI{50}{\milli\meter}, espa�ador \diameter \SI{12}{\milli\meter}, duas barras ferro \num{40}x\SI{5}{\milli\meter} furadas em ambas as pontas&\\
    & 1 & Guia do trav�o & Pe�a feita sob medida que engata no fixador trav�o--piv� e corre ao longo do suporte dos sensores de fim de curso e os ativa&\\
    & 5 & HIGLY VT16021C & Sensores de fim de curso de alavanca &\\
    & Indef. & Material fixa��o trav�o & Parafusos, rebites, anilhas, anilhas de mola e porcas de diversas medidas usados na fixa��o dos componentes do trav�o &\\
    & 1 & STM RMI S28 \todo{confirmar} & Redutor de parafuso sem fim e coroa, de rela��o de redu��o de \num{49} &\\
    & 1 & Polia 27 T 5 30 \cite{catalogo_polias} & Polia dentada passe m�trico "T5"{} -- DIN 7721--2, material aluminio UNI 9006 -- T6, \num{30} dentes, \SI{16}{\milli\meter} espessura &\\
    & 1 & Polia 27 T 5 60 \cite{catalogo_polias} & Polia dentada passe m�trico "T5"{} -- DIN 7721--2, material aluminio UNI 9006 -- T6, \num{60} dentes, \SI{16}{\milli\meter} espessura &\\
    & 1 & Correia T5 \SI{460}{\milli\meter} & Correia dentada de passe m�trico "T5"{}, de \SI{460}{\milli\meter} de per�metro &\\
    & 1 & Veio para redutor & Veio em lat�o, feito sob medida, para acoplar ao redutor da dire��o, e onde fixa a polia C\ref{componentes:C60}&\\
    & 1 & Engrenagem \num{120} dentes &  Engrenagem cil�ndrica, de m�dulo 1, �ngulo de press�o de 20�, material \texttt{C 43 UNI 7847}, de acordo com o cat�logo eurocorreias 2012 \cite{catalogo_eurocorreias} &\\
    & 1 & Engrenagem \num{12} dentes &  Engrenagem cil�ndrica, de m�dulo 1, �ngulo de press�o de 20�, material \texttt{C 43 UNI 7847}, de acordo com o cat�logo eurocorreias 2012 \cite{catalogo_eurocorreias} &\\
    & 1 & Potenci�metro \SI{5}{\kilo\ohm} & Potenci�metro de tr�s voltas, \SI{5}{\kilo\ohm} Vishay \todo{modelo?} &\\
    & 2 & Parafusos M4x35 & Parafusos para batente de fim de curso &\\
    & Indef. & Material fixa��o dire��o & Parafusos, porcas, anilhas, anilhas de mola, barras de metal e cavilhas de diversas medidas usados na fixa��o dos componentes da dire��o &\\
    & 1 & Sigmadrive PMT835M & Controlador de motor \ac{DC} de 4 quadrantes & \\
    & 1 & Agni B95-R & Motor \ac{DC} de �manes permanentes de fluxo axial & \\
    & 1 & Albright SW180B-12 & Contator de \ac{DC} de \SI{80}{\volt} &\\
    & 1 & Fus�vel CNN250 ANL& Fus�vel de encaixe ANL de \SI{250}{\ampere} &\\
    & 1 & Porta fus�vel ANL & Encaixe ANL para fus�vel & \\
    & 1 & Resist�ncia \SI{10}{\kilo\ohm} \SI{5}{\watt} & Resist�ncia de pr�-carga & \\
    & 1 & Acelerador punho & Acelerador de punho direito potenciom�trico, de \SI{5}{\kilo\ohm} & \\
    & 1 & Schneider ZB4 BS54 & Bot�o de emerg�ncia com contacto normalmente fechado & \\
    & 1 & Schneider ZB4 BZ105 & Contacto normalmente aberto para bot�o Schneider ZB4 BS54 &\\
    & 2 & Fus�veis \num{5x20} \SI{6.3}{\ampere} & Fus�veis de \SI{6.3}{\ampere}, de \SI{5x20}{mm} &\\
    & 2 & Fus�vel ESKA 522.717 & Fus�vel cer�mico de \SI{1}{\ampere}, de \SI{5x20}{mm}, I$^{2}$S=\num{1.10} &\\
    & 1 & Fus�vel \num{6x32} \SI{1}{\ampere} & Fus�vel de \SI{1}{\ampere}, de \SI{6x32}{mm} & H.morto\footnote{Dispositivo do homem morto}\\
    & 3 & Guita & Guita de $\approx$ \SI{3}{\meter} para acionamento do dispositivo do homem morto & H.morto\\
    & 1 & Encaixe para fus�vel \num{6x32} & Engate do dispositivo do homem morto & H.morto\\
    & 1 & Caixa de arruma��o & Caixa de arruma��o de pl�stico transparente, de cerca de \SI{25x25x50}{\centi\meter} com tampa &\\
    & 1 & Painel de instrumenta��o & Cantoneira em chapa para fixagem de instrumenta��o de painel & \\
    & 2 & Placas de acr�lico & Placas de acr�lico para fixa��o das placas eletr�nicas & \\
    & 1 & Dalf 1-F & Placa de microcontrolador & \ac{uC}\\
    & 1 & OSMC v4.24 & Ponte em H para comando de motores \ac{DC} &\\
    & 1 & Cabo de fita 10 fios & Cabo de fita, de 10 fios, com cerca de \SI{20}{\centi\meter}, f�mea--f�mea &\\
    & 1 & Cabo de fita 16 fios &  Cabo de fita, de 16 fios, com cerca de \SI{20}{\centi\meter}, com conetor f�mea numa ponta &\\
    & 4 & \ac{PCB} 115x160 & \ac{PCB} de \SI{115x160}{\milli\meter} &\\
    & 2 & \acp{LED} vermelhos & \acp{LED} \emph{through--hole} de \SI{12}{\volt} vermelhos & \\ 
    & 1 & \ac{LED} verde & \ac{LED} \emph{through--hole} de \SI{12}{\volt} verde & \\ 
    & 6 & D�odos 1N4001 & D�odos retificadores \emph{through--hole} &\\
    & 11 & D�odos 1N4002 & D�odos retificadores \emph{through--hole} &\\
    & 4 & R \SI{1.2}{\kilo\ohm} & Resist�ncias \emph{through--hole} \SI{1/4}{\watt} &\\
    & 1 & R \SI{1.5}{\kilo\ohm} \SI{5}{\watt} & Resist�ncia de pot�ncia \emph{through--hole} & \\ 
    & 3 & R \SI{2.7}{\kilo\ohm} & Resist�ncias \emph{through--hole} \SI{1/4}{\watt} &\\
    & 2 & R \SI{4.7}{\kilo\ohm} & Resist�ncias \emph{through--hole} \SI{1/4}{\watt} &\\
    & 1 & R \SI{7.5}{\kilo\ohm} & Resist�ncia \emph{through--hole} \SI{1/4}{\watt} &\\
    & 1 & R \SI{8.15}{\kilo\ohm} & Resist�ncia \emph{through--hole} \SI{1/4}{\watt} &\\
    & 1 & R \SI{10}{\kilo\ohm} & Resist�ncia \emph{through--hole} \SI{1/4}{\watt} &\\
    & 10 & R \SI{27}{\kilo\ohm} & Resist�ncias \emph{through--hole} \SI{1/4}{\watt} &\\
    & 1 & R \SI{33}{\kilo\ohm} & Resist�ncia \emph{through--hole} \SI{1/4}{\watt} &\\
    & 2 & R \SI{40}{\kilo\ohm} & Resist�ncias \emph{through--hole} \SI{1/4}{\watt} &\\
    & 15 & R \SI{47}{\kilo\ohm} & Resist�ncias \emph{through--hole} \SI{1/4}{\watt} &\\
    & 1 & R \SI{75}{\kilo\ohm} & Resist�ncia \emph{through--hole} \SI{1/4}{\watt} &\\
    & 1 & R \SI{220}{\kilo\ohm} & Resist�ncia \emph{through--hole} \SI{1/4}{\watt} &\\
    & 1 & C \SI{0.01}{\micro\farad} \SI{100}{\volt} & Condensador pol�mero \emph{through--hole}&\\
    & 1 & C \SI{0.01}{\micro\farad} & Condensador eletrol�tico \emph{through--hole}&\\
    & 3 & C \SI{20}{\nano\farad} & Condensadores eletrol�ticos \emph{through--hole}&\\
    & 6 & C \SI{0.1}{\micro\farad} & Condensadores eletrol�ticos \emph{through--hole}&\\
    & 3 & C \SI{20}{\micro\farad} & Condensadores eletrol�ticos \emph{through--hole}&\\
    & 2 & C \SI{22}{\micro\farad} & Condensadores eletrol�ticos \emph{through--hole}&\\
    & 3 & C \SI{47}{\micro\farad} & Condensadores eletrol�ticos \emph{through--hole}&\\
    & 1 & LM741 & Amplificador operacional \emph{through--hole} &\\
    & 1 & NE555 & Temporizador \emph{through--hole} de uso gen�rico &\\
    & 2 & 2N2219A & \acp{TJB} NPN \emph{through--hole} de elevada corrente &\\
    & 2 & BC547 & \acp{TJB} NPN \emph{through--hole} para comuta��o e amplifica��o &\\
    & 2 & BC557 & \acp{TJB} PNP \emph{through--hole} para comuta��o e amplifica��o &\\
    & 6 & Schrack RT424012 & Rel�s \ac{PCB} \ac{DPDT} de \SI{12}{\volt} \SI{8}{\ampere}&\\
    & 3 & Omron G2R-2 & Rel�s \ac{PCB} \ac{DPDT} de \SI{12}{\volt} &\\
    & 1 & Interruptor chave & Interruptor com chave unipolar \SI{1}{\ampere} de painel & \\
    & 2 & Bot�es de press�o & Bot�es unipolares \ac{SPST} \SI{1}{\ampere} de painel & \\
    & 3 & Interruptores alavanca & Interruptores de alavanca de painel \ac{DPDT} \emph{ON--ON} \SI{3}{\ampere} &\\
    & 1 & Interruptor alavanca & Interruptor de alavanca de painel \ac{SP3T} \emph{ON--OFF--ON} \SI{3}{\ampere} &\\
    & Indef. & Manga termoretr�til & Manga termoretr�til de v�rias dimens�es & \\
    & Indef. & Cabo de corrente preto & Cabo el�trico de elevado calibre \todo{qto?}, preto & \\ 
    & Indef. & Cabo de corrente vermelho & Cabo el�trico de elevado calibre \todo{qto?}, vermelho & \\ 
    & Indef. & Fio de microfone & Fio para microfone, blindado & \\
    & Indef. & Fio el�trico diverso & Fio el�trico de diversos calibres e cores, unifilar e multifilar, usado nas liga��es el�tricas & \\
    & Indef. & Etiquetas autocolantes & Etiquetas autocolantes &\\
    & Indef. & Bornes diversos & Bornes de diversas medidas e feitios usados nas liga��es el�tricas & \\
    & 3 & Fichas microfone & Pares de fichas f�mea e macho blindadas, para cabo de microfone & \\
    & 1 & Ficha 16 pinos & Ficha Molex \ac{SMD} Micro Fit 3.0 f�mea, de 16 pinos &\\
    & 1 & Ficha 6 pinos & Ficha Molex \ac{SMD} Micro Fit 3.0 f�mea, de 6 pinos &\\
    \bottomrule
\end{row_labeled_longtable}
%\end{longtable}
%\end{center}
%\end{table}
\cleardoublepage

		\cleardoublepage
	\end{appendix}
\end{appendices}

\phantomsection     %http://tex.stackexchange.com/questions/44088/when-do-i-need-to-invoke-phantomsection
\pagenumbering{gobble}
\addcontentsline{toc}{chapter}{Adenda}    %Add adenda to toc
\cleardoublepage    %http://tex.stackexchange.com/questions/9191/when-do-i-need-invoke-clearpage-manually
\section*{Adenda}
\label{sec:adenda}

(Adicione aqui as suas adendas ao documento.)
\begin{comment}
\begin{displayquote}Quem vier atr�s que feche a porta\end{displayquote}
    � melhor fazer algo mal feito, que deixar por fazer. Se este documento n�o f�r atualizado com o desenrolar do projecto, chegar-se-� a um ponto em que quem vier a seguir n�o conseguir� assimilar o que est� feito.
    Mas, caso n�o tenhas tempo, sintas que n�o tens conhecimento para o fazer ou simplesmente n�o te apete�a atualizar este manual como deve ser, usa esta adenda: adiciona aqui uma breve explica��o das atualiza��es que fizeste e as suas implica��es no texto desta manual, assim como a data, o teu nome e outra informa��o que achares relevante e n�o te preocupes com a formata��o.
\end{comment}


\todos  %list of things to do - should be the last todo item on the document
\end{document}
