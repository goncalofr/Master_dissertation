
\section{Obter ajuda}
\label{sec:ajuda}
O opera��o do m�dulo T2D do ROVIM pode suscitar d�vidas para os utilizadores mais inexperi�ntes. Existem no entanto v�rias formas de obter ajuda e esclarecer d�vidas:
\begin{comment}
    rascunho do m�todo de obten��o de ajuda
    -consulta do pessoal envolvido no projecto. As pessoas que trabalharam pr�viamente no projecto podem ajudar a esclarecer d�vidas ou dar conselhos sobre a utiliza��o do rob�t. Consulte a lista de colaboradores do projeto ROVIM, e do m�dulo \ref{sec:colab}{T2D} para mais detalhes.
    -consulta das teses e manuais dos v�rios componentes. Cada m�dulo do ROVIM originou uma tese de mestrado que pode ser consultada. Para o m�dulo T2D, a consulta dos [datasheets] dos v�rios componentes pode ser muito �til.
    -consulta do codigo: O \ref{}{c�digo} do controlador do m�dulo T2D pode ser �til no esclarecimento de d�vidas.
\end{comment}
\\
