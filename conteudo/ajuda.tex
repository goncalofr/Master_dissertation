
\section{Obter ajuda}
\label{sec:ajuda}
A utiliza��o do \ac{ROVIM} pode suscitar d�vidas a alguns utilizadores mais inexperientes. Existem no entanto v�rias formas de obter ajuda e esclarecer d�vidas ao seu dispor:
\begin{itemize}
    \item \textbf{Este manual.} Aqui s�o esclarecidas a maior parte das d�vidas que podem surgir na utiliza��o do ve�culo. Cont�m informa��o mais atualizada que as teses de mestrado.
    \item \textbf{Documenta��o produzida pelos anteriores colaboradores.} Esta deve conter detalhes sobre o c�digo fonte, esquemas el�tricos e \emph{layout} dos componentes eletr�nicos, entre outra informa��o. Esta documenta��o � de livre acesso, mas pode ser obtida junto dos colaboradores do projeto.
    \item \textbf{Documenta��o dos v�rios componentes individuais.} Muita desta documenta��o est� livremente acess�vel na internet, mas pode ser obtida junto dos colaboradores do projeto. A lista de componentes est� dispon�vel no \nameref{ap:c}.
    \item \textbf{Reposit�rio do c�digo do projeto,} em \url{https://github.com/ROVIM-T2D/ROVIM-T2D-Brain.git}. O c�digo aqui encontrado pode ser mais recente que o c�digo programado no ve�culo.
    \item \textbf{Os \nameref{sec:colaboradores} do projecto}. Os antigos e atuais colaboradores estar�o dispon�veis para ajudar a esclarecer d�vidas e aconselhar os atuais colaboradores e utilizadores.
\end{itemize}
