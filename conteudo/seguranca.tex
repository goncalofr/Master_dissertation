
\section{Considera\c{c}\~oes de seguran\c{c}a}
\label{sec:seguranca}
O ROVIM foi projectado com um �nfase na seguran�a dos utilizadores. A�nda ssim, � um prot�tipo insuficientemente refinado e testado para poder ser usado em condi��es desfavor�veis, ou por utilizadores inexperientes ou impreparados. Esta fragilidade aliada ao peso e pot�ncia do rob�t tornam a sua utiliza��o pot�ncialmente perigosa. Este cap�tulo imp�e aos utilizadores normas que devem ser seguidas constantemente e impreter�velmente para min�mizar os riscos e a gravidade de pot�nciais acidentes.
\begin{comment}
    rascunho das normas:
    -operar em terreno pouco acidentado
    -operar sem chuva
    -manter a frente do ve�culo sempre desimpedida de obst�culos
    -projectar a trajectoria pretendida para o ve�culo antes de o ligar, e garantir que esta se encontra desimpedida de pessoas ou outros obst�culos
    -alertar as pessoas na vizinha�a do ve�culo para o facto de este estar a ser utilizado e os seus perigos
    -l�r e compreender este manual antes de operar o ve�culo pela primeira vez
    -levantar as rodas de tr�s do ch�o quando o desligar
    -aprender a operar o veiculo com as 4 rodas no ar, antes de passar para o ch�o. Familiazirar-se com todas as funcionalidades e procedimentos de seguran�a
    -Manter sempre o dispositivo do homem-morto pronto a disparar (de prefer�ncia atado ao bra�o do utilizador)
\end{comment}
\\
