
\fancychapter{Ap�ndice A - c�digo do programa}
\label{ap:a}

\lstset{ %
language=[ANSI]C,                % choose the language of the code
basicstyle=\footnotesize,       % the size of the fonts that are used for the code
numbers=left,                   % where to put the line-numbers
numberstyle=\footnotesize,      % the size of the fonts that are used for the line-numbers
stepnumber=1,                   % the step between two line-numbers. If it is 1 each line will be numbered
numbersep=5pt,                  % how far the line-numbers are from the code
backgroundcolor=\color{white},  % choose the background color. You must add \usepackage{color}
showspaces=false,               % show spaces adding particular underscores
showstringspaces=false,         % underline spaces within strings
showtabs=false,                 % show tabs within strings adding particular underscores
frame=single,           % adds a frame around the code
%tabsize=2,          % sets default tabsize to 2 spaces
captionpos=b,           % sets the caption-position to bottom
breaklines=true,        % sets automatic line breaking
breakatwhitespace=false    % sets if automatic breaks should only happen at whitespace
%escapeinside={\%*}{*)}          % if you want to add a comment within your code
}

O c�digo fonte usado na programa��o da placa de controlo da \ac{T2D} � demasiado extenso para ser aqui listado, pelo que � fornecido num ficheiro complementar a esta disserta��o. Uma c�pia exata destes ficheiros, assim como outros ficheiros usados no projeto est� acess�vel em: \url{https://github.com/ROVIM-T2D/ROVIM-T2D-Brain/tree/\swversion}.

Os ficheiros \verb!main.c! e \verb!dalf.h! s�o altera��es aos ficheiros originais fornecidos com a placa Dalf para esta aplica��o, e s�o licenciados pela \emph{Embedded Electronics, LLC}\@.

A biblioteca \verb!dalf.lib! � tamb�m necess�ria para gerar o programa, mas n�o � listada aqui por n�o ter sido alterada da vers�o original, por estar em formato bin�rio, e por regras de licenciamento restritas. A licen�a destes ficheiros pode ser consultada no site do fornecedor, em: \url{http://www.embeddedelectronics.net/documents/Ver160/EULA.pdf}.

\done\todo{mostrar restantes ficheiros de c�digo}

\done\todo[Fix]{a listagem do c�digo ocupa atualmente ~70 p�ginas}

Os ficheiros listados no ficheiro s�o:

\begin{itemize}
    \item \verb!main.c!
    \item \verb!dalf_ext.c!
    \item \verb!dalf.h!
    \item \verb!config.h!
    \item \verb!p18f6722.h!
    \item \verb!rovim_t2d.c!
    \item \verb!rovim_t2d.h!
    \item \verb!rovim.h!
    \item \verb!rovim_config_t2d_development.h!
\end{itemize}

%\lstinputlisting{recursos/codigo/\swversion/main.c}
%\lstinputlisting{recursos/codigo/\swversion/dalf_ext.c}
%\lstinputlisting{recursos/codigo/\swversion/dalf.h}
%\lstinputlisting{recursos/codigo/\swversion/config.h}
%\lstinputlisting{recursos/codigo/\swversion/p18f6722.h}
%\lstinputlisting{recursos/codigo/\swversion/rovim_t2d.c}
%\lstinputlisting{recursos/codigo/\swversion/rovim_t2d.h}
%\lstinputlisting{recursos/codigo/\swversion/rovim.h}
%\lstinputlisting{recursos/codigo/\swversion/rovim_config_t2d_development.h}

\cleardoublepage
