%https://www.sharelatex.com/blog/2013/08/29/tikz-series-pt3.html
%http://tex.stackexchange.com/questions/249531/how-to-adjust-float-options-of-tikz-picture
%fluxogrammin'
\documentclass[tikz]{standalone}
%this flowchart is too big for one vertical page, so it has to be slipt.

\begin{document}
%configure the tikz styles used for the flowchart
%fyi: the figure never gets smaller than "text width" and "minimum height"
%configuração fluxogramas
\tikzstyle{startstop} = [rectangle, rounded corners, minimum height=1cm,text centered, text width=3cm, draw=black, fill=red!30]
\tikzstyle{process} = [rectangle, text centered, minimum height=1cm, text width=3cm, draw=black, fill=orange!30]
\tikzstyle{decision} = [diamond, minimum height=0.8cm, text badly centered, inner sep=1pt, text width=5em, draw=black, node distance=3cm, fill=green!30]
\tikzstyle{io} = [trapezium, trapezium left angle=70, trapezium right angle=110, minimum height=1cm, text badly centered,trapezium stretches body, text width=3cm, draw=black, fill=blue!30]
\tikzstyle{connector} = [circle, minimum height=0.5cm, text badly centered, text width=2em, draw=black, fill=brown!30]
\tikzstyle{arrow} = [thick,->,>=stealth]

%http://tex.stackexchange.com/questions/106306/using-footnotemark-with-letters
%So, footnotes in floats suck big time. There are two decent workarounds: threeparttable and minipage.
\renewcommand*{\thefootnote}{\alph{footnote}}
    \begin{minipage}[b]{\textwidth}
    \begin{center}
    \begin{tikzpicture}[node distance=2cm]
    %node description
        \node (inicio) [startstop] {Em \emph{lockdown}};
    \node (comando) [process, below of=inicio] {Enviar comando};%\footnotemark};
        \node (pronto?) [decision, below of=comando] {Erro detetado?};
    \node (permissao_destravar) [io, below of=pronto?, node distance=3cm] {Programa confirma};%\footnotemark};
        \node (corrigir) [process, right of=pronto?, node distance=4cm] {Corrigir};
        \node (destravar) [process, below of=permissao_destravar] {Pressionar bot�o de destravar at� acender Led\ref{leds:Led1}};
        \node (travao_destrava) [io, below of=destravar] {Trav�o destrava};
        \node (travou?) [decision, below of=travao_destrava] {Trav�o volta a travar?\footnotemark[1]};
        \node (contactor_arma) [io, below of=travou?, node distance=3cm] {Contatores armam};
        \node (pronto) [startstop, below of=contactor_arma] {Pronto};
    %edge description
        \draw [arrow] (inicio) -- (comando);
        \draw [arrow] (comando) -- (pronto?);
        \draw [arrow] (pronto?) -- node[anchor=east]{n�o} (permissao_destravar);
        \draw [arrow] (pronto?) -- node[anchor=south]{sim} (corrigir);
        \draw [arrow] (corrigir) |- (comando);
        \draw [arrow] (permissao_destravar) -- (destravar);
        \draw [arrow] (destravar) -- (travao_destrava);
        \draw [arrow] (travao_destrava) -- (travou?);
        \draw [arrow] (travou?) -- node[anchor=east]{n�o}(contactor_arma);
        \draw [arrow] (travou?) -- node[anchor=south]{sim} ++ (-4,0) |-  (inicio);
        %\path (pronto?)   --++  (4,0) node [inicio] {sim};
        \draw [arrow] (contactor_arma) -- (pronto);
    \end{tikzpicture}
    %\footnotetext{Comando para sair de \emph{lockdown}. Na interface s�rie este � o comando \texttt{G11}.}
    %\footnotetext{Na interface s�rie esta confirma��o � uma mensagem a instruir  o utilizador para proceder � destravagem.}
    \footnotetext[1]{O trav�o volta a travar ap�s soltar o bot�o de destravagem se tiver passado o limite de 20s para a destravagem, ou se ocorrer outro erro durante a destravagem}
\end{center}
\end{minipage}
\end{document}
