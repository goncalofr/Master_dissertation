%page layout learning:
% https://en.wikibooks.org/wiki/LaTeX/Page_Layout
\hoffset 0in    %this is a page layout dimension
\voffset 0in    %this is a page layout dimension

%Alternative set of page geometry
%\oddsidemargin 0.71cm
%\evensidemargin 0.04cm
%\marginparsep 0in
%\topmargin -0.25cm
%\textwidth 15cm
%\textheight 23.5cm

% geometry package provides flexible and easy interface to page dimensions
%define absolute margin values, ignoring individual page layout parameter dimensions
\usepackage[top=2.5cm, bottom=2.5cm, inner=2.9cm, outer=2.5cm]{geometry}

\usepackage{fancyhdr} % costumize page layout
\pagestyle{fancy} %set the fancy style of fancyhdr package (activate the package)
% Some info on latex marks to explain the following commands
%Marks usage: markboth{left head}{right head} \markright{right head}
%
%The \markboth and \markright commands are used in conjunction with the page style myheadings for setting either both or just the right heading. In addition to their use with the myheadings page style, you can use them to override the normal headings in the headings style, since LaTeX uses these same commands to generate those heads. You should note that a left-hand heading is generated by the last \markboth command before the end of the page, while a right-hand heading is generated by the first \markboth or \markright that comes on the page if there is one, otherwise by the last one before the page. 
%The \leftmark contains the Left argument of the Last \markboth on the page, the \rightmark
%contains the Right argument of the fiRst \markboth or the only argument of the fiRst \markright
%on the page. If no marks are present on a page they are “inherited” from the previous page.
%You can influence how chapter, section, and subsection information (only two of them!) is displayed
%by redefining the \chaptermark, \sectionmark, and \subsectionmark commands 4 . You must
%put the redefinition after the first call of \pagestyle{fancy} as this sets up the defaults.
%Let us illustrate this with chapter info. It is made up of three parts:
%• the number (say, 2), displayed by the macro \thechapter
%• the name (in English, Chapter), displayed by the macro \chaptername
%• the title, contained in the argument of \chaptermark.
% So, after redefining, it would look like this:
\renewcommand{\chaptermark}[1]{\markboth{\thechapter.\ #1}{}}
\renewcommand{\sectionmark}[1]{\markright{\thesection\ #1}}
\fancyhf{} %clear the header and footer; other the elements of the default "plain" pagestyle will appear.
%\fancyhead[LE]{\bfseries\nouppercase{\leftmark}}
%\fancyhead[RO]{\bfseries\nouppercase{\rightmark}}
\fancyfoot[LE,RO]{\bfseries\small\thepage} %Set page number (small font, bold [\bfseries]) on Left even page and on Right Odd page 
\renewcommand{\headrulewidth}{0.0pt}    %Set the footer and header line to 0 pt - make it disappear
\renewcommand{\footrulewidth}{0.0pt}
\addtolength{\headheight}{2pt} % make space for the rule
%Previous commands redefine the defaults set by the package fancyhdr.

%define/redefine custom pagestyles
%redefine the plain \pagestyle{}
\fancypagestyle{plain}{% Used in blank pages
   \fancyhead{} % get rid of headers
   \renewcommand{\headrulewidth}{0pt} % and the line
   \renewcommand{\footrulewidth}{0pt}
   \fancyfoot[LE,RO]{\bfseries\small\thepage} %set pagination as above
}

%define the begin pagestyle - for the preamble
\fancypagestyle{begin}{%
   \fancyhead{}
   \renewcommand{\headrulewidth}{0pt}
   \renewcommand{\footrulewidth}{0pt}
   \fancyfoot[LE,RO]{\bfseries\small\thepage}
}
%define the document pagestyle - for the main body of the document
%the lonely comment symbol (%) on the next line is to prevent unwanted whitespace
\fancypagestyle{document}{%Fancy head and foot with lines
	\fancyhf{} 
    %Set the header of Left Even and Rigt Odd margins to bold and make sure it is not spelled as uppercase (See fsncyhdr.pdf, section 9) - usefull for bibliography, for example.
	\fancyhead[LE]{\bfseries\nouppercase{\leftmark}}
	\fancyhead[RO]{\bfseries\nouppercase{\rightmark}}
	\fancyfoot[LE,RO]{\bfseries\small\thepage}
	%\renewcommand{\headrulewidth}{0pt}
	%\renewcommand{\footrulewidth}{0pt}
	\addtolength{\headheight}{2pt} % make space for the rule
}
%define the documentsimple pagestyle
\fancypagestyle{documentsimple}{% Without the fancy head and foot
	\fancyhf{}
	\fancyfoot[LE,RO]{\bfseries\small\thepage}
	%\renewcommand{\headrulewidth}{0pt}
	%\renewcommand{\footrulewidth}{0pt}
	\addtolength{\headheight}{2pt} % make space for the rule
}
%limit the section and toc depth to 5 levels
\setcounter{secnumdepth} {5}
\setcounter{tocdepth} {5}
%Set a subsubsection number to include also the subsection number.
% ex. subsection.subsubsection
\renewcommand{\thesubsubsection}{\thesubsection.\Alph{subsubsection}}

%{subfigure} package is now obsolete and was replace in this document by
%the {subcaption} package. This change removed the following 4 commands.
%% Length from the top of the subfigure box to the begining of the FIGURE
%% box.  Also from the bottom of the CAPTION to the bottom of the subfigure.
%\renewcommand{\subfigtopskip}{0.3 cm}
%space between the subcaption and the edge of the box
%\renewcommand{\subfigbottomskip}{0.2 cm}
%free space between the figure and the caption, if it exists
%\renewcommand{\subfigcapskip}{0.3 cm}
%indentation of the subfigure caption, in relation to the subfigure margins
%\renewcommand{\subfigcapmargin}{0.2 cm}

\graphicspath{{recursos/}}
