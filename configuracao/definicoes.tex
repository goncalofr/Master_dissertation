\typeout{}
\typeout{--------------------------------------------------------------}
\typeout{ +---+ Thesis Template                            }
\typeout{ +---+      Version 2.0, August 2011                         }
\typeout{ +---+  for Instituto Superior Tecnico (IST),                 }
\typeout{ +---+  Universidade Tecnica de Lisboa                         }
\typeout{ * Using Thesis Style form Pedro Tomas                                }
\typeout{ * Created to write Dissertations                             }
\typeout{ * Conforms with IST Master Degree format and with most important packages setup        }
\typeout{ * Should conform with IST PhD Degree format (not verified)   }
\typeout{                                                              }
\typeout{ AUTHOR: Miguel Amador and Joao Marques                                          }
\typeout{ MODIFIED BY: Goncalo Andre                                   }
\typeout{                                                              }
\typeout{ Important: Use all files in the archive, since this is based in all them. Modify dummy files at wish.                                        }
\typeout{--------------------------------------------------------------}
\typeout{}

% Defines an additional alphabet... not required in most cases
% ------------------------------------------------------------
% \DeclareMathAlphabet{\mathpzc}{OT1}{pzc}{m}{it}

% PACKAGE babel:
% ---------------
% The 'babel' package may correct some hyphenisation issues of latex. 
% However in most situations it is not required.
\usepackage[portuguese]{babel}

% PACKAGE fontenc:
% -----------------
% chooses T1-fonts and allows correct automatic hyphenation.
% from: http://tex.stackexchange.com/questions/664/why-should-i-use-usepackaget1fontenc/677#677
% If you don't use \usepackage[T1]{fontenc},
%   -Words containing accented characters cannot be automatically hyphenated,
%   -You cannot properly copy-and-paste such words from the output (DVI/PS/PDF)
%   -Characters like the pipe sign, less than and greater sign give unexpected results in text.
\usepackage[T1]{fontenc}
% http://tex.stackexchange.com/questions/44694/fontenc-vs-inputenc
% allows the user to input accented characters directly form the keyboard
\usepackage[latin1]{inputenc}
%\usepackage{lmodern}

%package verbatim - needed for comment environment
\usepackage{verbatim}

% Package ulem (underlining for emphasis).
% the ulem package provides various types of underlining that can
% stretch between words and be broken across lines.
\usepackage{ulem} % Allows the use of other text emphatizer commands
\normalem % () defines \emph{} back to italic, instead of underline (reverts change from ulem package). 
\raggedbottom %declaration makes all pages the height of the text on that page. No extra vertical space is added. The \flushbottom declaration makes all text pages the same height, adding extra vertical space when necessary to fill out the page.

% PACKAGE date time:
% -----------------
% Lets you alter the format of the date that \today returns.
\usepackage{datetime}
\newdateformat{todaythesis}{%
\monthname[\THEMONTH]  \THEYEAR}

% PACKAGE latexsym:
% -----------------
% Defines additional latex symbols. May be required for thesis with many math symbols.
\usepackage{latexsym}

% PACKAGE amsmath, amsthm, amssymb, amsfonts:
% -------------------------------------------
% This package is typically required. Among many other things it adds the possibility
% to put symbols in bold by using \boldsymbol (not \mathbf); defines additional 
% fonts and symbols; adds the \eqref command for citing equations. I prefer the style
% "(x.xx)" for referering to an equation than to use "equation x.xx".
\usepackage{amsmath, amsthm, amssymb, amsfonts, amsbsy}

% PACKAGE multirow, colortbl, longtable:
% ---------------------------------------
% These packages are most usefull for advanced tables. The first allows to join rows 
% throuhg the command \multirow which works similarly with the command \multicolumn
% The second package allows to color the table (both foreground and background)
% The third package is only required when tables extend beyond the length of one page;
% with compatibilities with the tabular environment. The last allow the definitions of landscape pages, allowing the use of a different orientation for wider graphics or tables. See package documentation to see the implementation.
\usepackage{multirow}
\usepackage{colortbl}
\usepackage{supertabular}
\usepackage{pdflscape}
% \usepackage{longtable}

% PACKAGE graphics, epsfig, subfigure, caption:
% ---------------------------------------------
% Packages for figures... well you will certainly need these packages, with the exception
% of the 'caption' package. This only allows to define extra caption options.
% Notice that subfigure allows to place figures within figures with its own caption. It
% should be avoided to create an eps file with subfigures. That will mean that you won't be 
% able to reference those subfigures. Instead create an EPS file (the only graphics format supported
% by latex) for each of the subfigures and then use the command \subfigure (see below).
\usepackage{graphics}
\usepackage{graphicx}
\usepackage{epsfig}
%\usepackage[hang,small,bf]{subfigure} - obsolete package, replace by subcaption
%\usepackage[footnotesize,bf,center]{caption}
\usepackage{dcolumn}
\usepackage{bm}
\usepackage{booktabs}
\usepackage{rotating}
\usepackage{multirow}

\usepackage[font=small,labelfont=bf,textfont=normalfont]{caption}
\usepackage[font=small,labelfont=bf,textfont=normalfont]{subcaption}
%http://tex.stackexchange.com/questions/91566/syntax-similar-to-centering-for-right-and-left/91580#91580?newreg=ab93b1a4f79a4315b773d13d37d37678
%easilly define horizontal orientation of images.
\usepackage[export]{adjustbox}

% Improves the interface for defining floating objects such as figures and tables
\usepackage{float}

% PACKAGE algorithmic, algorithm
% ------------------------------
% These packages are required if you need to describe an algorithm.
% \usepackage{algorithmic}
% \usepackage[chapter]{algorithm}

% PACKAGE natbib/cite
% -------------------
% The two packages are not compatible, and you should use one of the two. Notice however that the
% IEEE BiBTeX stylesheet is imcompatible with the natbib package. If using the IEEE format, use the 
% cite package instead
\usepackage[square,numbers,sort&compress]{natbib}
%\usepackage{cite}

% PACKAGE acronyum
% -----------------
% This package is most useful for acronyms. The package guarantees that all acronyms definitions are 
% given at the first usage. IMPORTANT: do not use acronyms in titles/captions; otherwise the definition 
% will appear on the table of contents.
\usepackage[printonlyused]{acronym}
\usepackage[titletoc,title,header]{appendix} %modify the titles of appendixes
%Set page numbering by chapter. The noauto option tells the package not to set pagenumbers: this has to be done by the user
\usepackage[noauto]{chappg}

% PACKAGE extra_functions VER COMO DEVE SER
% -----------------
% My Personal package: defines the following commands:
% \fancychapter{chaptername) -> Prints a fancier chapter (you can also use the fancychapter package for this)
% \hline{width} -> use for a replacement of the \hline command
% \Mark1, \Mark2, \Mark3, ...
\usepackage{extra_functions}


% PACKAGE hyperref
% -----------------
% Set links for references and citations in document
% Some MiKTeX distributions have faulty PDF creators in which case this package will not work correctly
% Long live Linux :D
% This package should come to the end of the includes, because it redefines many macros
\usepackage[plainpages=false]{hyperref}
% configure the hyperref package
\hypersetup{
             colorlinks=false, %color the links
             citecolor=red, %colour of citation links
             breaklinks=true,   %allow links to break over lines
             bookmarksnumbered=true, %put section numbers in bookmarks
             bookmarksopen=true, %open up bookmark three
             debug=true, %extra diagnostic messages on the log file
             % pdf metadata
             pdftitle={Thesis Title},
             pdfauthor={Author Name},
             pdfsubject={Master Thesis in Biomedical Engineering},
             pdfcreator={Document Creator Name},
             pdfkeywords={Template, Latex, Thesis}}

%Manage symbols the same way we manage acronyms
%\usepackage[final]{listofsymbols}
\usepackage{symlist}

% Set paragraph counter to alphanumeric mode
\renewcommand{\theparagraph}{\Alph{paragraph}~--}

\newcommand{\figref}[1]{Figure \ref{#1}}
\newcommand{\equationref}[1]{Equation (\ref{#1})}
\newcommand{\tableref}[1]{Table (\ref{#1})}

\newcommand{\textreg}{$\textsuperscript{\textregistered}$}
