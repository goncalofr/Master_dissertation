\begin{resumo}

Os avan�os t�cnol�gicos da �ltima d�cada permitiram um crescimento nas aplica��es militares de ve�culos aut�nomos. Com o objetivo de explorar o seu pot�ncial na vigil�ncia de instala��es militares, a Academia Militar encomendou o desenvolvimento e constru��o de um prot�tipo funcional de um ve�culo aut�nomo. Nesta disserta��o os sistemas de tra��o, travagem e dire��o desse prot�tipo s�o abordados. A literatura atual foca-se em aplica��es comerciais de ve�culos rodovi�rios, onde a autonomia � o principal problema. No entanto, equipas n�o profissionais lidam antes disso com dificuldades em capturar uma vis�o clara do projecto e fracos procedimentos de seguran�a.
Uma moto-quatro com um m�dulo de baterias el�tricas e atuadores embarcados para os sistemas da tra��o, travagem e dire��o, assim como os mecanismos de controlo e interface, � aqui proposta e avaliada, numa abordagem focada na flexibilidade de desenho e na seguran�a dos utilizadores. As limita��es da solu��o proposta s�o identificadas e s�o propostas corre��es.

\end{resumo}
